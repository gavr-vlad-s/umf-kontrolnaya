\documentclass[12pt]{article}

%%
%% Uncomment the line \usepackage{literat}
%% if you are going to use
%% Literaturnaya font instead of LH font
%%
%% \usepackage{literat}
%%
%% You may comment the line \usepackage{mathtext}
%% if your text does not have russian letters
%% in math formulae
\usepackage{ccaption}
\usepackage{verbatim}
\usepackage{amsfonts}
\usepackage{amssymb}
\usepackage{amsmath}
\usepackage{indentfirst}
%%
%% If your text is in english
%% then you should substitute
%% the option `russian' by `english'
%%
\usepackage{graphicx}

\usepackage[utf8]{inputenc}
\usepackage[russian]{babel}
\textheight 24.0cm
\textwidth 16cm
%\renewcommand{\baselinestretch}{1.347}
\newcommand{\No}{${\cal N}{\!}\underline{\circ}$}
\voffset -2cm
\hoffset .0cm
\oddsidemargin 0.5mm
\evensidemargin 0.5mm
\topmargin -0.4mm
\righthyphenmin=2
\hfuzz=12.7pt
\makeatletter
%\renewcommand\section{\@startsection {section}{1}{\z@}%
%                                   {3.5ex \@plus 1ex \@minus .2ex}%
%                                   {2.3ex \@plus.2ex}%
%                                   {\normalfont\Large\bfseries}}
\renewcommand{\thesection}{\arabic{section}}
\@addtoreset{equation}{section}
\@addtoreset{figure}{section}
%\@addtoreset{table}{section}
\renewcommand{\theequation}{\thesection.\arabic{equation}}
\renewcommand{\thefigure}{\thesection.\arabic{figure}}
%\renewcommand{\thetable}{\thesection.\arabic{table}}
\makeatother
\newcommand{\dom}{\rm dom}
\newcounter{rem}[section]
\renewcommand{\therem}{\thesection.\arabic{rem}}
\newenvironment{Remark}{\par\refstepcounter{rem}
	\bf Замечание \therem. \sl}{\rm\par}

\renewcommand{\theenumi}{\arabic{enumi}}
\renewcommand{\labelenumi}{\theenumi)}
\newcommand{\udc}[1]{УДК #1}

\newcounter{lem}[section]
\renewcommand{\thelem}{\thesection.\arabic{lem}}
\newenvironment{Lemma}{\par\refstepcounter{lem}\bf Лемма \thelem. \sl}{\rm\par}

\newcounter{cor}[section]
\renewcommand{\thecor}{\thesection.\arabic{cor}}
\newenvironment{Corrolary}{\par\refstepcounter{cor}\bf Следствие \thecor. \sl}{\rm\par}
\newcounter{theor}[section]
\renewcommand{\thetheor}{\thesection.\arabic{theor}}
\newenvironment{Theorem}{\par\refstepcounter{theor}\bf Теорема \thetheor. \sl}{\rm\par}
%\let \kappa=\ae
\newcommand{\diag}{\mathop{\rm diag}}
\newcommand{\epi}{\mathop{\rm epi}}
\newenvironment{Proof}{\par\noindent\bf Доказательство.\rm}{ \par}

\newcounter{exam}[section]
\renewcommand{\theexam}{\thesection.\arabic{exam}}
\newenvironment{Example}{\par\refstepcounter{exam}\bf Пример \theexam. \sl}{\rm\par}

\newcounter{prob}[section]
\renewcommand{\theprob}{\thesection.\arabic{prob}}
\newenvironment{Problem}{\par\refstepcounter{prob}\bf №\theprob. \sl}{\rm\par}

\newcounter{sol}[section]
\renewcommand{\thesol}{\thesection.\arabic{sol}}
\newenvironment{Solution}{\par\refstepcounter{sol}\bf Решение. \rm}{\rm\par}

\newcounter{defin}[section]
\renewcommand{\thedefin}{\thesection.\arabic{defin}}
\newenvironment{Definition}{\par\refstepcounter{defin}\bf Определение
	\thedefin.\sl}{\rm\par}

\newcounter{answ}[section]
\renewcommand{\theansw}{\thesection.\arabic{answ}}
\newenvironment{Answer}{\par\refstepcounter{answ}\theansw. \rm}{\rm\par}

\newcounter{exerc}[section]
\renewcommand{\theexerc}{\thesection.\arabic{exerc}}
\newenvironment{Exercise}{\par\refstepcounter{exerc}\bf Упражнение
	\theexerc.\it}{\rm\par}
\newcommand{\ljoq}{<<}
\newcommand{\rjoq}{>>}
\newcommand{\vraisup}{\mathop{\rm vraisup}}
\newcommand{\pr}{\mathop{\rm pr}}
\newcommand{\sgn}{\mathop{\rm sgn}}


\captiondelim{. }

\setcounter{page}{3}


\makeatletter
\def\@seccntformat#1{\csname the#1\endcsname.\quad}
\makeatother

%\setcounter{section}{1}
%\setcounter{section}{1}
\begin{document}
%
%\vspace*{50ex}
\large

\vspace*{5cm}

\renewcommand{\contentsname}{}

\centerline{\textbf{Содержание}}

\vspace*{1cm}

\tableofcontents

\newpage

\addcontentsline{toc}{section}{Введение}
\section*{Введение}

\newpage

\section{Вариант 1}
\begin{Problem}
Используя формулу Даламбера, найти решение задачи
\begin{equation*}
u_{tt}=u_{xx}+\sin t,\,\,\,u|_{t=0}=x,\,\,\,u_t|_{t=0}=x.
\end{equation*}
\end{Problem}

\begin{Problem}
Определить решение начальной задачи для однородного волнового уравнения в точке $x=\frac{\pi}{2}$. Начальные функции имеют вид
$$
\varphi(x)=\left\{
\begin{array}{ll}
\sin x,& |x|<\pi,\\
0,     & |x|>\pi;
\end{array}
\right.
\psi(x)=\left\{
\begin{array}{ll}
v_0,   & |x|<\pi,\\
0,     & |x|>\pi.
\end{array}
\right.
$$
\end{Problem}
	
\begin{Problem}
Построить профиль полуограниченной струны с жёстко закреплённым концом $x=0$ в момент времени $t=\frac{5c}{2a}$, если начальное отклонение отлично от нуля только на интервале $(c,\,4c)$ и
имеет форму ломаной с вершинами в точках $(c,\,0)$, $(2c,\,2h)$, $(3c,\,\frac{3h}{2})$, $(4c,\,0)$. 

Начальная скорость равна нулю. Найти формулы, представляющие закон движения точки $x=\frac{5c}{2}$.
\end{Problem}
	
\begin{Problem}
Полуограниченной струне со свободным концом $x=0$ в начальный момент времени $t=0$ с помощью поперечного удара передаётся импульс $I$ в точках $x=x_0$ и $x=4x_0$. Найти отклонения точек струны в момент времени $t=\frac{3x_0}{2a}$.
\end{Problem}
	
\begin{Problem}
Найти решение начально--краевой задачи
\begin{gather*}
u_{tt}-4u_{xx}=0,\,\,\,t>0,\,\,\,x>0;\\
u|_{t=0}=2-x,\,\,\,u_t|_{t=0}=2,\\
(u_t+3u_x)|_{x=0}=3t-e^t.
\end{gather*}
\end{Problem} 
	
\begin{Problem}
Решить задачу о колебаниях струны, один конец которой ($x=0$) свободен, а другой ($x=\pi$) --- закреплён жёстко. Начальное отклонение  и начальная скорость имеют вид:
$$
u|_{t=0}=\cos\frac{x}{2},\,\,\,u_t|_{t=0}=\cos\frac{x}{2}.
$$
\end{Problem}
	
\begin{Problem}
Рассмотреть задачу о поперечных колебаниях струны, закреплённой на конце $x=0$ и подверженной на конце $x=l$ действию силы $Asin\omega t$. Начальные условия --- нулевые. Найти решение при всех $0<t<\frac{3l}{2a}$.
\end{Problem}
	
\section{Вариант 2}
\begin{Problem}
Используя формулу Даламбера, найти решение задачи
\begin{equation*}
u_{tt}=u_{xx}+\cos t,\,\,\,u|_{t=0}=x,\,\,\,u_t|_{t=0}=0.
\end{equation*}
\end{Problem}

\begin{Problem}
Определить решение начальной задачи для однородного волнового уравнения в точке $x=\frac{\pi}{4}$. Начальные функции имеют вид
$$
\varphi(x)=\left\{
\begin{array}{ll}
\cos x,& |x|<\frac{\pi}{2},\\
0,     & |x|>\frac{\pi}{2};
\end{array}
\right.
\psi(x)=\left\{
\begin{array}{ll}
v_0,   & |x|<\frac{\pi}{2},\\
0,     & |x|>\frac{\pi}{2}.
\end{array}
\right.
$$
\end{Problem}

\begin{Problem}
Построить профиль полуограниченной струны со свободным концом $x=0$ в момент времени $t=\frac{3c}{2a}$, если начальное отклонение отлично от нуля только на интервале $(c,\,4c)$ и
имеет форму ломаной с вершинами в точках $(c,\,0)$, $(2c,\,\frac{3h}{2})$, $(3c,\,2h)$, $(4c,\,0)$.

Начальная скорость равна нулю. Найти формулы, представляющие закон движения точки $x=\frac{5c}{2}$.
\end{Problem}
	
\begin{Problem}
Полуограниченной струне с жёстко закреплённым концом $x=0$ в начальный момент времени $t=0$ с помощью поперечного удара передаётся импульс $I$ в точках $x=x_0$ и $x=3x_0$. Найти отклонения точек струны в момент времени $t=\frac{3x_0}{2a}$.
\end{Problem}
	
\begin{Problem}
Найти решение начально--краевой задачи
\begin{gather*}
4u_{tt}-u_{xx}=0,\,\,\,t>0,\,\,\,x>0;\\
u|_{t=0}=\cos x,\,\,\,u_t|_{t=0}=0,\\
u|_{x=0}=1+\sin t.
\end{gather*}
\end{Problem} 
	
\begin{Problem}
Решить задачу о колебаниях струны, один конец которой ($x=0$) свободен, а другой ($x=\pi$) --- закреплён жёстко. Начальное отклонение  и начальная скорость имеют вид:
$$
u|_{t=0}=\cos\frac{x}{2},\,\,\,u_t|_{t=0}=0.
$$
\end{Problem}
	
\begin{Problem}
Рассмотреть задачу о поперечных колебаниях струны, один конец которой ($x=0$) двигается по заданному закону $u|_{x=0}=A\sin \omega t$, а другой ($x=l$) --- свободен. 
Начальные условия --- нулевые. Найти решение при всех $0<t<\frac{3l}{2a}$.
\end{Problem}

\section{Вариант 3}
\begin{Problem}
Используя формулу Даламбера, найти решение задачи
\begin{equation*}
u_{tt}=u_{xx}+xt,\,\,\,u|_{t=0}=x,\,\,\,u_t|_{t=0}=\sin x.
\end{equation*}
\end{Problem}

\begin{Problem}
Определить решение начальной задачи для однородного волнового уравнения в момент времени $t=\frac{\pi}{4a}$. Начальные функции имеют вид
$$
\varphi(x)=\left\{
\begin{array}{ll}
\sin x,& |x|<\pi,\\
0,     & |x|>\pi;
\end{array}
\right.
\psi(x)=\left\{
\begin{array}{ll}
v_0,   & |x|<\pi,\\
0,     & |x|>\pi.
\end{array}
\right.
$$
\end{Problem}

\begin{Problem}
Построить профиль полуограниченной струны со свободным концом $x=0$ в момент времени $t=\frac{3c}{2a}$, если начальное отклонение отлично от нуля только на интервале $(c,\,4c)$ и
имеет форму ломаной с вершинами в точках $(c,\,0)$, $(2c,\,2h)$, $(3c,\,2h)$, $(4c,\,0)$.

Начальная скорость равна нулю. Найти формулы, представляющие закон движения точки $x=\frac{5c}{2}$.
\end{Problem}

\begin{Problem}
Полуограниченной струне с жёстко закреплённым концом $x=0$ в начальный момент времени $t=0$ с помощью поперечного удара передаётся импульс $I$ в точках $x=\frac{x_0}{2}$ и $x=x_0$. Найти отклонения точек струны в момент времени $t=\frac{3x_0}{2a}$.
\end{Problem}

\begin{Problem}
Найти решение начально--краевой задачи
\begin{gather*}
9u_{tt}-u_{xx}=0,\,\,\,t>0,\,\,\,x>0;\\
u|_{t=0}=\sin x,\,\,\,u_t|_{t=0}=1,\\
u|_{x=0}=\sin t.
\end{gather*}
\end{Problem} 

\begin{Problem}
Решить задачу о колебаниях струны, один конец которой ($x=0$) закреплён жёстко, а другой ($x=\pi$) --- свободен. Начальное отклонение  и начальная скорость имеют вид:
$$
u|_{t=0}=\sin\frac{x}{2},\,\,\,u_t|_{t=0}=0.
$$
\end{Problem}

\begin{Problem}
Рассмотреть задачу о поперечных колебаниях струны, свободной на конце $x=0$ и подверженной на конце $x=l$ действию силы $A\sin\omega t$. 
Начальные условия --- нулевые. Найти решение при всех $0<t<\frac{3l}{2a}$.
\end{Problem}

\section{Вариант 4}
\begin{Problem}
Используя формулу Даламбера, найти решение задачи
\begin{equation*}
u_{tt}=u_{xx}+xt,\,\,\,u|_{t=0}=x,\,\,\,u_t|_{t=0}=\sin x.
\end{equation*}
\end{Problem}

\begin{Problem}
Определить решение начальной задачи для однородного волнового уравнения в момент времени $t=\frac{\pi}{4a}$. Начальные функции имеют вид
$$
\varphi(x)=\left\{
\begin{array}{ll}
\cos x,& |x|<\frac\pi2,\\
0,     & |x|>\frac\pi2;
\end{array}
\right.
\psi(x)=\left\{
\begin{array}{ll}
v_0,   & |x|<\frac\pi2,\\
0,     & |x|>\frac\pi2.
\end{array}
\right.
$$
\end{Problem}

\begin{Problem}
Построить профиль полуограниченной струны со свободным концом $x=0$ в момент времени $t=\frac{5c}{2a}$, если начальное отклонение отлично от нуля только на интервале $(c,\,4c)$ и
имеет форму ломаной с вершинами в точках $(c,\,0)$, $(2c,\,h)$, $(3c,\,h)$, $(4c,\,0)$.

Начальная скорость равна нулю. Найти формулы, представляющие закон движения точки $x=\frac{5c}{2}$.
\end{Problem}

\begin{Problem}
Полуограниченной струне с жёстко закреплённым концом $x=0$ в начальный момент времени $t=0$ с помощью поперечного удара передаётся импульс $I$ в точках $x=x_0$ и $x=2x_0$. Найти отклонения точек струны в момент времени $t=\frac{3x_0}{2a}$.
\end{Problem}

\begin{Problem}
Найти решение начально--краевой задачи
\begin{gather*}
u_{tt}-4u_{xx}=0,\,\,\,t>0,\,\,\,x>0;\\
u|_{t=0}=x^2,\,\,\,u_t|_{t=0}=2,\\
(u_t+3u_x)|_{x=0}=2.
\end{gather*}
\end{Problem} 

\begin{Problem}
Решить задачу о колебаниях струны, оба конца которой ($x=0$, $x=\pi$) --- свободны. Начальное отклонение  и начальная скорость имеют вид:
$$
u|_{t=0}=\cos x,\,\,\,u_t|_{t=0}=0.
$$
\end{Problem}

\begin{Problem}
Рассмотреть задачу о поперечных колебаниях струны со свободным концом $x=0$, если на конце $x=l$ задано смещение $u|_{x=l}=A\sin\omega t$. 
Начальные условия --- нулевые. Найти решение при всех $0<t<\frac{3l}{2a}$.
\end{Problem}

\section{Вариант 5}
\begin{Problem}
Используя формулу Даламбера, найти решение задачи
\begin{equation*}
u_{tt}=u_{xx}+2t,\,\,\,u|_{t=0}=x,\,\,\,u_t|_{t=0}=\sin x.
\end{equation*}
\end{Problem}

\begin{Problem}
Определить решение начальной задачи для однородного волнового уравнения в момент времени $t=\frac{1}{a}$. Начальные функции имеют вид
$$
\varphi(x)=\left\{
\begin{array}{ll}
x^2-2x,& 0<x<2,\\
0,     & x<0 \text{ или } x>2;
\end{array}
\right.
\psi(x)=\left\{
\begin{array}{ll}
v_0,   & 0<x<2,\\
0,     & x<0 \text{ или } x>2.
\end{array}
\right.
$$
\end{Problem}

\begin{Problem}
Построить профиль полуограниченной струны с жёстко закреплённым концом $x=0$ в момент времени $t=\frac{3c}{2a}$, если начальное отклонение отлично от нуля только на интервале $(c,\,4c)$ и
имеет форму ломаной с вершинами в точках $(c,\,0)$, $(2c,\,2h)$, $(3c,\,2h)$, $(4c,\,0)$.

Начальная скорость равна нулю. Найти формулы, представляющие закон движения точки $x=\frac{5c}{2}$.
\end{Problem}

\begin{Problem}
Полуограниченной струне со свободным концом $x=0$ в начальный момент времени $t=0$ с помощью поперечного удара передаётся импульс $I$ в точках $x=l$ и $x=2l$. Найти отклонения точек струны в момент времени $t=\frac{l}{2a}$.
\end{Problem}

\begin{Problem}
Найти решение начально--краевой задачи
\begin{gather*}
u_{tt}-u_{xx}=0,\,\,\,t>0,\,\,\,x>0;\\
u|_{t=0}=x+1,\,\,\,u_t|_{t=0}=1,\\
u|_{x=0}=\cos t.
\end{gather*}
\end{Problem} 

\begin{Problem}
Решить задачу о колебаниях струны, один конец которой ($x=0$) закреплён жёстко, а другой ($x=\pi$) --- свободен. Начальное отклонение  и начальная скорость имеют вид:
$$
u|_{t=0}=\sin\frac{x}2,\,\,\,u_t|_{t=0}=\sin\frac{x}2.
$$
\end{Problem}

\begin{Problem}
Рассмотреть задачу о поперечных колебаниях струны, свободной на конце $x=l$ и подверженной на конце $x=0$ действию силы $A\sin\omega t$. 
Начальные условия --- нулевые. Найти решение при всех $0<t<\frac{3l}{2a}$.
\end{Problem}

\section{Вариант 6}

\section{Вариант 7}

\section{Вариант 8}

\section{Вариант 9}

\section{Вариант 10}

\section{Вариант 11}

\section{Вариант 12}

\section{Вариант 13}

\section{Вариант 14}

\section{Вариант 15}

\section{Вариант 16}

\section{Вариант 17}

\section{Вариант 18}

\section{Вариант 19}

\section{Вариант 20}

\section{Вариант 21}

\section{Вариант 22}

\section{Вариант 23}

\section{Вариант 24}

\section{Вариант 25}
\end{document}