\documentclass[12pt]{article}

%%
%% Uncomment the line \usepackage{literat}
%% if you are going to use
%% Literaturnaya font instead of LH font
%%
%% \usepackage{literat}
%%
%% You may comment the line \usepackage{mathtext}
%% if your text does not have russian letters
%% in math formulae
\usepackage{ccaption}
   \usepackage{verbatim}
   \usepackage{amsfonts}
   \usepackage{amssymb}
   \usepackage{amsmath}
    \usepackage{indentfirst}
%%
%% If your text is in english
%% then you should substitute
%% the option `russian' by `english'
%%
   \usepackage{graphicx}

   \usepackage[utf8]{inputenc}
   \usepackage[russian]{babel}
\textheight 24.0cm
\textwidth 16cm
%\renewcommand{\baselinestretch}{1.347}
\newcommand{\No}{${\cal N}{\!}\underline{\circ}$}
\voffset -2cm
\hoffset .0cm
\oddsidemargin 0.5mm
\evensidemargin 0.5mm
\topmargin -0.4mm
\righthyphenmin=2
\hfuzz=12.7pt
\makeatletter
%\renewcommand\section{\@startsection {section}{1}{\z@}%
%                                   {3.5ex \@plus 1ex \@minus .2ex}%
%                                   {2.3ex \@plus.2ex}%
%                                   {\normalfont\Large\bfseries}}
\renewcommand{\thesection}{\arabic{section}}
\@addtoreset{equation}{section}
\@addtoreset{figure}{section}
%\@addtoreset{table}{section}
\renewcommand{\theequation}{\thesection.\arabic{equation}}
\renewcommand{\thefigure}{\thesection.\arabic{figure}}
%\renewcommand{\thetable}{\thesection.\arabic{table}}
\makeatother
\newcommand{\dom}{\rm dom}
\newcounter{rem}[section]
\renewcommand{\therem}{\thesection\arabic{rem}}
\newenvironment{Remark}{\par\refstepcounter{rem}
\bf Замечание \therem. \sl}{\rm\par}

\renewcommand{\theenumi}{\arabic{enumi}}
\renewcommand{\labelenumi}{\theenumi)}
\newcommand{\udc}[1]{УДК #1}

\newcounter{lem}[section]
\renewcommand{\thelem}{\thesection\arabic{lem}}
\newenvironment{Lemma}{\par\refstepcounter{lem}\bf Лемма \thelem. \sl}{\rm\par}

\newcounter{cor}[section]
\renewcommand{\thecor}{\thesection\arabic{cor}}
\newenvironment{Corrolary}{\par\refstepcounter{cor}\bf Следствие \thecor. \sl}{\rm\par}
\newcounter{theor}[section]
\renewcommand{\thetheor}{\thesection\arabic{theor}}
\newenvironment{Theorem}{\par\refstepcounter{theor}\bf Теорема \thetheor. \sl}{\rm\par}
%\let \kappa=\ae
\newcommand{\diag}{\mathop{\rm diag}}
\newcommand{\epi}{\mathop{\rm epi}}
\newenvironment{Proof}{\par\noindent\bf Доказательство.\rm}{ \par}

\newcounter{exam}[section]
\renewcommand{\theexam}{\thesection\arabic{exam}}
\newenvironment{Example}{\par\refstepcounter{exam}\bf Пример \theexam. \sl}{\rm\par}

\newcounter{prob}[section]
\renewcommand{\theprob}{\thesection\arabic{prob}}
\newenvironment{Problem}{\par\refstepcounter{prob}\bf Задача \theprob. \sl}{\rm\par}

\newcounter{sol}[section]
\renewcommand{\thesol}{\thesection\arabic{sol}}
\newenvironment{Solution}{\par\refstepcounter{sol}\bf Решение. \rm}{\rm\par}

\newcounter{defin}[section]
\renewcommand{\thedefin}{\thesection\arabic{defin}}
 \newenvironment{Definition}{\par\refstepcounter{defin}\bf Определение
\thedefin.\sl}{\rm\par}

\newcounter{answ}[section]
\renewcommand{\theansw}{\thesection\arabic{answ}}
\newenvironment{Answer}{\par\refstepcounter{answ}\theansw. \rm}{\rm\par}

\newcounter{exerc}[section]
\renewcommand{\theexerc}{\thesection\arabic{exerc}}
 \newenvironment{Exercise}{\par\refstepcounter{exerc}\bf Упражнение
\theexerc.\it}{\rm\par}
\newcommand{\ljoq}{<<}
\newcommand{\rjoq}{>>}
\newcommand{\vraisup}{\mathop{\rm vraisup}}
\newcommand{\pr}{\mathop{\rm pr}}
\newcommand{\sgn}{\mathop{\rm sgn}}


\captiondelim{. }

\setcounter{page}{3}
%\setcounter{section}{1}
\begin{document}
%
%\vspace*{50ex}
\large
\unitlength=5mm

\vspace*{5cm}

\renewcommand{\contentsname}{}

\centerline{\textbf{Содержание}}

\vspace*{1cm}

\tableofcontents

\newpage


\addcontentsline{toc}{section}{Введение}
\section*{Введение}
%Данное учебно--методическое пособие является продолжением части I, одноимённого пособия \cite{Fourier1},
%в котором рассматриваются задачи для функций с двумя независимыми переменными. Цель части II ---
%научить студентов решать избранные многомерные задачи методом разделения переменных. Из всего многообразия
%задач математической физики в пособии выделены три типа задач: начально--краевые (смешанные) задачи
%для уравнения колебания мембраны и уравнения теплопроводности, а также краевые задачи для уравнения
%Пуассона, возникающие при нахождении стационарной температуры в однородном теле. Эти задачи рассматриваются
%в случаях, когда пространственные переменные изменяются в областях, ограниченных окружностями или круговыми цилиндрами.
%На одном из этапов решения таких задач методом разделения переменных возникает задача Штурма--Лиувилля
%для уравнения Бесселя, что приводит к использованию цилиндрических функций. Основные сведения о цилиндрических
%функциях и решения задач Штурма--Лиувилля для уравнения Бесселя даны в \cite{BesselMet}. Поэтому в этом
%пособии приводятся лишь необходимые результаты решения, а именно, собственные числа и собственные функции
%соответствующих задач Штурма--Лиувилля, с указанием формул пособия \cite{BesselMet}. Все рассматриваемые
%в пособии многомерные задачи обладают осевой симметрией и сводятся к одномерным. При подборе задач
%использовались книги \cite{arsenin}--\cite{LebedevSkalskayaUflyand}.
\section{Сведения из теории формальных языков}
Данный раздел посвящён минимально необходимым для реализации лексического анализа сведениям из теории формальных языков и конечных автоматов. 
	\subsection{Определение алфавитов и языков}
Прежде всего приведём определение алфавита и языка.

\textbf{Алфавитом} называется любое конечное множество некоторых символов. При этом понятие символа не определяется, поскольку оно в теории формальных языков является базовым.
 
Как правило, алфавит будем обозначать заглавными греческими буквами (например, буквой $\Sigma$), возможно, с нижними индексами. 
 
Приведём примеры алфавитов:
\begin{enumerate}
\item $\{0,1\}$ --- алфавит $\Sigma_1$, состоящий из нуля и единицы;

\item $\{A,B,\dots,Z\}$ --- алфавит $\Sigma_2$, состоящий из заглавных латинских букв;

\item $\{\text{А},\text{Б},\text{В},\text{Г},\text{Д},\text{Е},\text{Ё},\dots,\text{Я}\}$ --- алфавит $\Sigma_3$, состоящий из заглавных русских букв;

\item 
$\{$\textbf{int}, \textbf{void}, \textbf{return}, *, '(', ')', '\{', '\}', ';', main, \textit{number}$\}$ --- алфавит $\Sigma_4$, состоящий из ключевых слов \textbf{int}, \textbf{void}, \textbf{return} языка C, идентификатора main, звёздочки, круглых скобок, фигурных скобок, точки с запятой, и целых чисел \textit{number} (синтаксис целых чисел --- как в языке C);

\item $\{a_1,a_2,a_3,a_4\}$ --- алфавит $\Sigma_5$, состоящий из каких--то четырёх символов.
\end{enumerate}

Из символов алфавита можно составлять \textbf{строки}, то есть конечные последовательности символов. Если строка состоит из символов алфавита $\Sigma$, то её называют \textbf{строкой над алфавитом $\Sigma$}. \textbf{Длиной строки} $x$ называется количество символов в этой строке. Длину строки $x$ будем обозначать $|x|$. Строка, вообще не содержащая символов, называется \textbf{пустой строкой} и будет обозначаться $\varepsilon$.

Приведём примеры строк:
\begin{enumerate}
\item 0111001 --- строка над алфавитом $\Sigma_1$;

\item ENGLISH, INTEL --- строки над алфавитом $\Sigma_2$;

\item МОСКВА, ГОРЬКИЙ, АЛЁШКОВО --- строки над алфавитом $\Sigma_3$;

\item \textbf{int} main (\textbf{void})'\{' \textbf{return} \textit{number}';' '\}' ---
строка над алфавитом $\Sigma_4$;

\item $a_1a_3a_2a_2a_4$ --- строка над алфавитом $\Sigma_5$. 
\end{enumerate}

Далее потребуется операция \textbf{сцепления} (иногда говорят \textbf{конкатенации}) строк. Эта операция заключается в приписывании одной строки в конец другой. Например, если строки $\alpha$ и $\beta$ таковы, что $\alpha=\text{abc}$, $\beta=\text{defg}$,
то конкатенация строк $\alpha$ и $\beta$ обозначается $\alpha\beta$, и представляет собой строку $\text{abcdefg}$.

Множество всех строк над алфавитом $\Sigma$ обозначается $\Sigma^*$. Скажем, если $\Sigma=\{0,1\}$, то $\Sigma^*=\{\varepsilon,0,1,00,01,10,11,000,\dots,1010,\dots\}$. Ясно, что множество всех строк над заданным алфавитом --- бесконечно, а точнее --- счётно.

Любой набор строк над некоторым алфавитом называется \textbf{языком} (ещё называют \textbf{формальным языком}, чтобы отличать от естественных языков). Допустим, из всевозможных строк над алфавитом $\{0,1,2,\\ 3,4,5,6,7,8,9,'.',-\}$ можно выбрать те, которые являются корректной записью некоторого вещественного числа: $L=\{0,-1.5,1002.123345,777,\\ \dots\}$. Языки обозначаются заглавными латинскими буквами, возможно с нижними индексами. Язык может являться конечным множеством строк: если $L$ --- язык над алфавитом $\{a,b\}$, содержащий лишь строки короче трёх символов, то $L=\{\varepsilon,a,b,aa,ab,ba,bb\}$.
	\subsection{Операции над языками и регулярные выражения}

Поскольку язык --- это некоторое множество строк, то нужно уметь это множество как--то описывать. Одним из способов описания являются так называемые регулярные выражения. Прежде чем определить, что такое регулярное выражение, нужно определить операции над языками. Операции над языками, которые нам потребуются, собраны в приводимой ниже табл.\ref{langops!table}.	\newpage


\begin{table}[!h]\label{langops!table}	
\centering
\vspace{1mm}
\caption{Операции над языками.}
\begin{tabular}{|c|p{70mm}|}\hline
Операция 		                                      & Определение и обозначение операции \\ \hline
Объединение $L$ и $M$                              & $L\cup M=\{s: s\in L\text{ или }s\in M\}$   \\ \hline
Сцепление $L$ и $M$                                    & $LM=\{st: s\in L\text{ и }t\in M\}$                 \\ \hline
Замыкание Клин\'{и} языка $L$                & $L^*=\bigcup\limits_{i=0}^\infty L^i$         \\ \hline
Положительное замыкание языка $L$  & $L^+=\bigcup\limits_{i=1}^\infty L^i$         \\ \hline
\end{tabular}
\end{table}

В этой таблице  $L$ и $M$ --- некоторые языки 	
\section{Примеры реализации лексического анализа}
В настоящем разделе мы приводим примеры реализации лексического анализа для простых ситуаций.
%В пособии не приводится вывод уравнений и граничных условий. Соответствующий материал можно найти в
%\cite{arsenin}--\cite{Vladimirov}.
%
%\noindent
%$\bullet$\,\,\,\,\textbf{Задача о распространении тепла.}
%
%Уравнение, описывающее распространение тепла в изотропной однородной среде имеет вид
%\begin{equation}\label{HeatEquation}
%u_t-a^2\Delta u=\frac{f(x,t)}{c\rho},
%\end{equation}
%где $u(x,t)$ --- температура в точке $x\equiv(x_1,x_2,x_3)$ в момент времени $t$,
%$c$ --- коэффициент теплоёмкости материала, $\rho$ --- плотность материала, $f(x,t)$ --- плотность
%источников тепла, $a^2=\frac{k}{c\rho}$, $k$ --- коэффициент теплопроводности материала,
%$\Delta u\equiv u_{x_1x_1}+u_{x_2x_2}+u_{x_3x_3}$ --- трёхмерный оператор Лапласа.
%
%Для однозначного определения решения уравнения (\ref{HeatEquation}) в области $Q\subset R^3$ при $t>0$ необходимо задать начальную температуру
%в области $Q$,
%\begin{equation}\label{HeatEquation:initial}
%u|_{t=0}=u_0(x),\,\,\,x\in Q,
%\end{equation}
%и условие на границе области $Q$:
%\begin{gather}\label{HeatEquation:bc}
%\left.\left(\alpha\frac{\partial u}{\partial n}+\beta u\right)\right|_{\partial Q}=u_1(x,t),\,\,\,
%x\in\partial Q,\,\,\,t>0,
%\end{gather}
%где $\alpha$ и $\beta$ --- неотрицательные константы, одновременно не обращающиеся в нуль.
%
%Начально--краевая (смешанная) задача для уравнения (\ref{HeatEquation}), заключается в нахождении функции $u(x,t)$, определяемой в области $x\in\bar Q$, $t\geq0$, удовлетворяющей уравнению (\ref{HeatEquation}) в области $x\in Q$, $t>0$, начальному условию  (\ref{HeatEquation:initial}) и граничному условию
%(\ref{HeatEquation:bc}). Граница области $Q$, функции $f(x,t)$, $u_0(x)$, $u_1(x,t)$, постоянные $\alpha$, $\beta$, определяются физическими условиями  задачи, и называются входными данными задачи.
%
%Ниже перечислим частные случаи граничного условия (\ref{HeatEquation:bc}),
%сформулировав происходящий физический процесс:
%\begin{enumerate}
%    \item
%на границе области $Q$ поддерживается заданная температура:
%\begin{equation}\label{HeatEquation:boundary_temperature}
%u|_{\partial Q}=u_1(x,t),\,\,\,x\in\partial Q,\,\,\,t>0;
%\end{equation}
%    \item
%через границу области $Q$ подаётся тепловой поток плотности $q$:
%\begin{equation}\label{HeatEquation:teplovoj_potok}
%\left.k\frac{\partial u}{\partial n}\right|_{\partial Q}=q(x,t),\,\,\,x\in\partial Q,\,\,\,t>0;
%\end{equation}
%    \item
%граница области $Q$ теплоизолирована, т.е. тепловой поток через границу равен нулю:
%\begin{equation}\label{HeatEquation:teploizoljacia}
%\left.\frac{\partial u}{\partial n}\right|_{\partial Q}=0,\,\,\,t>0;
%\end{equation}
%    \item
%на границе $\partial Q$ происходит теплообмен по закону Ньютона со средой, имеющей температуру $u_1(x,t)$:
%\begin{equation}\label{Newton}
%\left.\frac{\partial u}{\partial n}\right|_{\partial Q}+H[u-u_1]|_{\partial Q}=0,\,\,\,t>0,
%\end{equation}
%где $H=\frac k\alpha$.
%\end{enumerate}
%
%
%
%\noindent
%$\bullet$\,\,\,\,\textbf{Задача о колебаниях мембраны.}
%
%Пусть в положении равновесия мембрана занимает область $Q$ плоскости $(x_1,x_2)$. Уравнение, описывающее поперечные колебания однородной мембраны имеет вид
%\begin{equation}\label{WaveEquation}
%u_{tt}-a^2\Delta u=\frac{f(x,t)}{\rho},
%\end{equation}
%где $u(x,t)$ --- отклонение мембраны от положения равновесия в точке $x\equiv(x_1,x_2)$ в момент времени $t$,
%$\rho$ --- плотность материала мембраны, $f(x,t)$ --- плотность внешней силы, распределённой по площади
%мембраны, $\Delta u\equiv u_{x_1x_1}+u_{x_2x_2}$ --- двумерный оператор Лапласа, $a^2=\frac{T}\rho$.
%
%
%Для однозначного описания процесса колебаний мембраны нужно задать  начальное отклонение и начальную скорость точек мембраны,
%\begin{equation}\label{WaveEquation:initial}
%u|_{t=0}=u_0(x),\,\,\,u_t|_{t=0}=u_1(x),\,\,\,x\in Q,
%\end{equation}
%а также условие на границе мембраны (граничное условие):
%\begin{gather}\label{WaveEquation:bc}
%\left.\left(\alpha\frac{\partial u}{\partial n}+\beta u\right)\right|_{\partial Q}=u_2(x,t),\,\,\,
%x\in \partial Q,\,\,\,t>0,
%\end{gather}
%где $\alpha$ и $\beta$ --- неотрицательные постоянные, одновременно не обращающиеся в нуль.
%
%Сформулируем начально--краевую задачу: найти функцию, определённую в области
%$x\in\bar Q$, $t\geq0$, удовлетворяющую уравнению (\ref{WaveEquation})
%в области $x\in Q$, $t>0$, начальным условиям  (\ref{WaveEquation:initial})
%и граничному условию (\ref{WaveEquation:bc}).
%
%Входными данными задачи о колебаниях мембраны являются область $Q\subset R^2$,
%функции $f(x,t)$, $u_0(x)$, $u_1(x)$, $u_2(x)$ и постоянные $\alpha$, $\beta$.
%
%Постоянные  $\alpha$ и $\beta$ определяются физическими условиями задачи. Приведём возможные варианты таких условий:
%\begin{enumerate}
%    \item
%край мембраны жёстко закреплён, т.е. смещения точек края мембраны равны нулю:
%\begin{equation}\label{WaveEquation:fixed}
%u|_{x\in\partial Q}=0,\,\,\,x\in\partial Q,\,\,\,t>0;
%\end{equation}
%    \item
%к краю мембраны приложена сила с плотностью $f_1(x,t)$:
%\begin{equation}\label{WaveEquation:force}
%\left.\frac{\partial u}{\partial n}\right|_{\partial Q}=\frac{f_1(x,t)}T
%,\,\,\,x\in\partial Q,\,\,\,t>0;
%\end{equation}
%    \item
%край мембраны свободен, т.е. может свободно перемещаться по вертикальной боковой поверхности цилиндра с основанием $\partial Q$:
%\begin{equation}\label{WaveEquation:free}
%\left.\frac{\partial u}{\partial n}\right|_{\partial Q}=0,\,\,\,t>0;
%\end{equation}
%    \item
%край мембраны упруго закреплён:
%\begin{equation}\label{WaveEquation:uprugo}
%\left.\left[\frac{\partial u}{\partial n}+Hu\right]\right|_{\partial Q}=0,\,\,\,x\in\partial Q,\,\,\,t>0,
%\end{equation}
%где $H=\frac kT$.
%\end{enumerate}
%
%Если колебания мембраны происходят в среде, оказывающей сопротивление,
%пропорциональное скорости, то уравнение колебаний мембраны имеет вид
%\begin{equation}
%u_{tt}+2\nu u_t-a^2\Delta u=\frac{f(x,t)}\rho,\,\,\,x\in Q,\,\,\,t>0.
%\end{equation}
%
%\noindent
%$\bullet$\,\,\,\,\textbf{Задача нахождения стационарной температуры.}
%
%Стационарной называется температура, которая не зависит от времени,
%а является функцией лишь пространственных переменных. Вследствие этого уравнение   (\ref{HeatEquation}) принимает
%вид уравнения Пуассона
%\begin{equation}\label{HeatEquation:st}
%\Delta u=-\frac{f(x)}{c\rho a^2}.
%\end{equation}
%
%
%Для уравнения (\ref{HeatEquation:st}) ставится краевая задача: найти решение уравнения (\ref{HeatEquation:st}) в области $Q$, удовлетворяющее на границе области $Q$ условию
%\begin{equation}
%\left.\left[\alpha(x)\frac{\partial u}{\partial n}+\beta(x)u\right]\right|_{\partial Q}=u_1(x),\,\,\,x\in\partial Q,
%\end{equation}
%где $\alpha(x)$ и $\beta(x)$ --- неотрицательные функции, одновременно не обращаю\-щиеся в ноль.
%
%Примеры граничных условий в задаче нахождения стационарной температуры те же, что и в нестационарном случае
%((\ref{HeatEquation:boundary_temperature})--(\ref{Newton})), но правые части граничных условий не зависят от $t$.
%\section{Метод разделения переменных для задач с однородными уравнениями и с однородными граничными условиями}
%\begin{Problem}\label{p21}
%Круглая закреплённая по краю мембрана радиуса $r_0$ с центром в начале координат находится
%в среде, сопротивление которой пропорционально скорости. Коэффициент
%пропорциональности $2\nu<\frac{2a\mu_1}{r_0}$, где $\mu_1$ --- первый положительный
%ноль функции Бесселя $J_0(\mu)$. Изучить свободные колебания мембраны, если
%известно, что при $t=0$ её форма имеет радиальный характер (является функцией
%расстояния от центра), а начальная скорость равна нулю.
%\end{Problem}
%\begin{Solution}В задаче ничего не сказано о внешней силе, распределённой по
%поверхности мембраны. По условию задачи колебания мембраны вызваны только
%начальным отклонением от положения равновесия. Поэтому запишем однородное волновое
%уравнение в среде с сопротивлением
%\begin{equation}\label{volnur2Dssopr}
%u_{tt}+2\nu u_t-a^2(u_{xx}+u_{yy})=0.
%\end{equation}
%В начальный момент времени известна форма мембраны
%\begin{equation}\label{nachformamembr}
%u|_{t=0}=u_0(r),\mbox{ где $r=\sqrt{x^2+y^2}$},
%\end{equation}
%и начальная скорость
%\begin{equation}\label{nacscormembr}
%u_t|_{t=0}=0.
%\end{equation}
%
%Кроме начальных условий следует задать граничное условие. Так как край мембраны
%жёстко закреплён, отклонений точек мембраны на границе не происходит и граничное
%условие имеет вид
%\begin{equation}\label{gruslmembr}
%u|_{x^2+y^2=r_0^2}=0.
%\end{equation}
%
%Сформулируем начально--краевую задачу. Найти функцию $u(x,y,t)$, определённую в области
%$x^2+y^2\leq r_0^2$, $t\geq0$, удовлетворяющую уравнению (\ref{volnur2Dssopr})
%в области $x^2+y^2< r_0^2$, $t>0$, начальным условиям (\ref{nachformamembr}), (\ref{nacscormembr}) и граничному условию
%(\ref{gruslmembr}).
%
%При решении задач методом Фурье переменные должны разделяться как в однородном уравнении,
%так и в однородных граничных условиях. В данной задаче переменные не разделяются в
%граничном условии (\ref{gruslmembr}). В этих случаях, как правило, используют другие
%независимые переменные, отражающие симметрию задачи. Граница области, в
%которой решается задача (окружность), является координатной линией полярной системы координат.
%Поэтому сделаем замену переменных
%$$
%\begin{cases}
%x=r\cos\varphi,\\
%y=r\sin\varphi.
%\end{cases}
%$$
%В результате получим начально--краевую задачу
%\begin{gather}\label{volnur2Dssoprpolsyst}
%u_{tt}+2\nu u_t-a^2\left[
%\frac1r
%\frac\partial{\partial r}
%\left(r\frac{\partial u}{\partial r}\right)+
%\frac1{r^2}\frac{\partial^2u}{\partial\varphi^2}\right]=0,\\
%\notag
%u|_{t=0}=u_0(r),\,\,\,%
%\notag
%u_t|_{t=0}=0,\\
%\notag
%u|_{r=r_0}=0.
%\end{gather}
%
%Входные данные в (\ref{volnur2Dssoprpolsyst}) (правые части уравнения,
%начальных и граничных условий) не зависят от $\varphi$. Поэтому решение
%задачи также не зависит от $\varphi$,
%\begin{gather*}
%u=u(r,t),\,\,\,\frac{\partial u}{\partial\varphi}=0,
%\end{gather*}
%что приводит к упрощению уравнения:
%\begin{gather}\label{volnur2Dssoprpolsyst:upr}
%u_{tt}+2\nu u_t-
%\frac{a^2}r
%\frac\partial{\partial r}
%\left(r\frac{\partial u}{\partial r}\right)=0,\,\,\,
%0<r<r_0,\,\,\,t>0,\\
%\label{volnur2Dssoprpolsyst:upr:otkl}
%u|_{t=0}=u_0(r),\,\,\,0\leq r\leq r_0,\\
%\label{volnur2Dssoprpolsyst:upr:scor}
%u_t|_{t=0}=0,\,\,\,0\leq r\leq r_0,\\
%\label{volnur2Dssoprpolsyst:upr:grusl}
%u|_{r=r_0}=0,\,\,\,t\geq0.
%\end{gather}
%
%Так как точка $r=0$ является особой точкой уравнения (\ref{volnur2Dssoprpolsyst:upr}),
%следует потребовать ограниченности решения в этой точке в любой момент времени
%\begin{gather}\label{volnur2Dssoprpolsyst:upr:grusl_ogr}
%|u|_{r=0}|<\infty,\,\,\,t\geq0.
%\end{gather}
%
%Для решения задачи (\ref{volnur2Dssoprpolsyst:upr})--(\ref{volnur2Dssoprpolsyst:upr:grusl_ogr})
%сначала рассмотрим вспомогательную задачу об отыскании частных решений уравнения (\ref{volnur2Dssoprpolsyst:upr}),
%удовлетворяющих граничным условиям (\ref{volnur2Dssoprpolsyst:upr:grusl}), (\ref{volnur2Dssoprpolsyst:upr:grusl_ogr}),
%отличных от нуля и имеющих специальный вид
%\begin{gather}\label{spsolution}
%u(r,t)=R(r)T(t).
%\end{gather}
%Здесь $R(r)$ --- функция, зависящая только от переменной $r$, а $T(t)$ ---
%функция, зависящая только от переменной $t$. Подставим (\ref{spsolution}) в
%уравнение (\ref{volnur2Dssoprpolsyst:upr}):
%\begin{gather*}
%T''(t)R(r)+2\nu T'(t)R(r)=\frac{a^2}rT(t)\frac{d}{dr}\left(r\frac{dR(r)}{dr}\right).
%\end{gather*}
%Поделив на $a^2T(t)R(r)$, получим равенство
%\begin{gather}\label{TeqR}
%\frac{T''(t)+2\nu T'(t)}{a^2T(t)}=\frac1{rR(r)}\frac{d}{dr}\left(r\frac{dR(r)}{dr}\right),
%\end{gather}
%которое должно выполняться при всех $0<r<r_0$, $t>0$. Особенностью равенства (\ref{TeqR})
%является то, что его левая часть не зависит от $r$, а правая не зависит от $t$.
%Такое возможно лишь в случае, когда выражения левой и правой частей не зависят
%от $r$ и $t$ и равны одной и той же постоянной. Обозначим эту постоянную $-\lambda$:
%\begin{gather*}
%\frac{T''(t)+2\nu T'(t)}{a^2T(t)}=\frac1{rR(r)}\frac{d}{dr}\left(r\frac{dR(r)}{dr}\right)=-\lambda.
%\end{gather*}
%Отсюда выводим
%\begin{gather}
%\label{Tequation}
%T''(t)+2\nu T'(t)+a^2\lambda T(t)=0,\\
%\label{Requation}
%\frac{d}{dr}\left(r\frac{dR(r)}{dr}\right)+\lambda rR(r)=0.
%\end{gather}
%
%Таким образом, уравнение  (\ref{volnur2Dssoprpolsyst:upr}) распалось
%на два обыкновенных дифференциальных уравнения, или, как говорят,
%в уравнении  (\ref{volnur2Dssoprpolsyst:upr}) разделились переменные.
%
%Подставляя произведение  (\ref{spsolution}) в однородные граничные условия
% (\ref{volnur2Dssoprpolsyst:upr:grusl}) и (\ref{volnur2Dssoprpolsyst:upr:grusl_ogr}),
%будем иметь
%\begin{gather}\label{bcR}
%|R(0)|<\infty,\,\,\,R(r_0)=0.
%\end{gather}
%Задача (\ref{Requation}), (\ref{bcR}) является примером задачи Штурма--Лиувилля. Те значения параметра
%$\lambda$, при которых задача имеет нетривиальные решения, называются собственными значениями, а
%соответствующие им нетривиальные решения --- собственными функциями. Решить задачу Штурма--Лиувилля ---
%значит найти все собственные значения и собственные функции. В данном пособии решение задач
%Штурма--Лиувилля не приводится, а указывается, где его можно найти.
%Выпишем собственные функции, их нормы и собственные числа задачи Штурма--Лиувилля
%(\ref{Requation}), (\ref{bcR})(\cite[стр.31--33]{BesselMet}):
%\begin{gather}\label{SchturmLiouvilleR:solution}
%R_n(r)=J_0\left(\frac{\mu_nr}{r_0}\right),\,\,\,
%\|R_n\|^2=\frac{r_0^2}2J_1^2(\mu_n),\\
%\notag
%\lambda_n=
%\left(\frac{\mu_n}{r_0}\right)^2,\,\,\,n=1,2,\dots;
%\end{gather}
%где $0<\mu_1<\mu_2<\dots<\mu_n<\dots$ --- положительные нули функции $J_0(\mu)$.
%
%Подставим собственные значения $\lambda_n$ в уравнение (\ref{Tequation}),
%а решения, соответствующие этим $\lambda_n$, обозначим $T_n(t)$:
%\begin{gather}\label{Tnequation}
%T''_n(t)+2\nu T'_n(t)+\left(\frac{a\mu_n}{r_0}\right)^2T_n(t)=0,\,\,\,n=1,2,\dots
%\end{gather}
%Уравнения (\ref{Tnequation}) являются линейными однородными с постоянными коэффициентами,
%соответствующие им характеристические уравнения имеют вид
%\begin{gather*}
%p^2_n+2\nu p_n+\left(\frac{a\mu_n}{r_0}\right)^2=0,\,\,\,n=1,2,\dots
%\end{gather*}
%Поскольку корни характеристического уравнения --- комплексно--соп\-ря\-жён\-ные числа
%$p_n=-\nu\pm i\omega_n$, где $\omega_n=\sqrt{\left(\frac{a\mu_n}{r_0}\right)^2-\nu^2}$,
%то общее решение дифференциального уравнения (\ref{Tnequation})
%имеет вид
%\begin{gather*}
%T_n(t)=e^{-\nu t}[C_n\cos(\omega_nt)+D_n\sin(\omega_nt)].
%\end{gather*}
%Согласно формуле  (\ref{spsolution}) запишем частные решения уравнения
%(\ref{volnur2Dssoprpolsyst:upr})
%\begin{gather*}
%u_n(r,t)=T_n(t)R_n(r)=
%e^{-\nu t}[C_n\cos(\omega_nt)+D_n\sin(\omega_nt)]J_0\left(\frac{\mu_nr}{r_0}\right)\!,\\
%\,\,\,n=1,2,\dots
%\end{gather*}
%В соответствии с принципом дискретной суперпозиции ряд, составленный из решений
%уравнения (\ref{volnur2Dssoprpolsyst:upr}),
%\begin{gather}\label{u:series}
%u(r,t)=\sum\limits_{n=1}^\infty u_n(r,t)=\\
%=\notag
%\sum\limits_{n=1}^\infty
%e^{-\nu t}[C_n\cos(\omega_nt)+D_n\sin(\omega_nt)]J_0\left(\frac{\mu_nr}{r_0}\right)
%\end{gather}
%также удовлетворяет (\ref{volnur2Dssoprpolsyst:upr}) при условии его равномерной сходимости и возможности
%почленного дифференцирования по $r$ и $t$ до второго порядка включительно. Поскольку каждое слагаемое ряда
% (\ref{u:series}) удовлетворяет граничным условиям (\ref{volnur2Dssoprpolsyst:upr:grusl}) и
% (\ref{volnur2Dssoprpolsyst:upr:grusl_ogr}), то им же будет удовлетворять и сумма ряда, если ряд сходится
% равномерно по $r$ и $t$  в области $0\leq r\leq r_0$, $t\geq0$.
%
%Подберём коэффициенты $C_n$ и $D_n$ так, чтобы функция, представляемая формальным
%рядом, удовлетворяла начальным условиям (\ref{volnur2Dssoprpolsyst:upr:otkl}),
%(\ref{volnur2Dssoprpolsyst:upr:scor}). Продифференцируем ряд (\ref{u:series}) по $t$:
%\begin{gather}\label{ut:series}
%u_t(r,t)=\sum\limits_{n=1}^\infty\{-\nu e^{-\nu t}[C_n\cos(\omega_nt)+D_n\sin(\omega_nt)]+\\
%\notag+e^{-\nu t}\omega_n[-C_n\sin(\omega_nt)+D_n\cos(\omega_nt)]\}J_0\left(\frac{\mu_nr}{r_0}\right)
%\end{gather}
%Подставляя (\ref{u:series}),  (\ref{ut:series}) в начальные условия
%(\ref{volnur2Dssoprpolsyst:upr:otkl}),
%(\ref{volnur2Dssoprpolsyst:upr:scor}), получим равенства
%\begin{gather*}
%\sum\limits_{n=1}^\infty
%C_nJ_0\left(\frac{\mu_nr}{r_0}\right)=u_0(r),\\
%\sum\limits_{n=1}^\infty
%[-\nu C_n+\omega_nD_n]J_0\left(\frac{\mu_nr}{r_0}\right)=0,
%\end{gather*}
%которые представляют собой разложение заданных функций $u_0(r)$ и $0$
%в виде рядов по собственным функциям $J_0\!\left(\frac{\mu_nr}{r_0}\right)$, $k=1,2,\dots$.
%В соответствии с теоремой Стеклова \cite[стр.7]{Fourier1}, заключаем, что
%\begin{gather}\label{TnCauchyProblem}
%C_n=\frac2{r_0^2J_1^2(\mu_n)}
%\int\limits_0^{r_0}\xi u_0(\xi)J_0\left(\frac{\mu_n\xi}{r_0}\right)d\xi\equiv\alpha_n,\\
%\notag
%-\nu C_n+\omega_nD_n=0.
%\end{gather}
%Определив из системы (\ref{TnCauchyProblem}) коэффициенты $C_n$, $D_n$ и подставив их в ряд
% (\ref{u:series}), найдём решение задачи:
%\begin{gather*}
%u(r,t)=e^{-\nu t}\sum\limits_{n=1}^\infty
%\alpha_n\left[\cos(\omega_nt)+\frac{\nu\sin(\omega_nt)}{\omega_n}\right]
%J_0\left(\frac{\mu_nr}{r_0}\right).
%\end{gather*}
%\end{Solution}
%
%\begin{Problem}
%Найти закон выравнивания заданного осесимметричного начального распределения
%температуры $u(r,0)=r^2$ в бесконечном цилиндре радиуса $r_0$, боковая поверхность
%которого теплоизолирована.
%\end{Problem}
%\begin{Solution}
%Запишем однородное уравнение теплопроводности, так как внутри
%бесконечного цилиндра нет источников тепла:
%\begin{gather}\label{heat:equation}
%u_t-a^2(u_{xx}+u_{yy}+u_{zz})=0.
%\end{gather}
%Известно начальное распределение температуры
%\begin{gather}\label{heat:initial}
%u|_{t=0}=r^2,\mbox{ где $r=\sqrt{x^2+y^2}$.}
%\end{gather}
%
%Кроме того, по условию задачи боковая поверхность цилиндра теплоизолирована.  Это означает, что
%тепловой поток через боковую поверхность равен нулю, то есть
%\begin{gather}\label{heat:boundary}
%\left.k\frac{\partial u}{\partial n}\right|_{r=r_0}=0,
%\end{gather}
%где $n$ --- единичный вектор внешней нормали к боковой поверхности цилиндра.
%
%Сформулируем начально--краевую задачу. Найти функцию $u(x,y,z,t)$,
%определённую в области $x^2+y^2\leq r_0^2$, $-\infty<z<\infty$, $t\geq0$,
%удовлетворяющую уравнению (\ref{heat:equation}) в области
% $x^2+y^2<r_0^2$, $-\infty<z<\infty$, $t>0$, начальному условию
% (\ref{heat:initial}) и граничному условию (\ref{heat:boundary}).
%
%Граница области, в которой решается задача,
%является координатной поверхностью цилиндрической системы координат.
%Поэтому используем цилиндрические координаты:
%\begin{gather*}
%  \begin{cases}
%    x=r\cos\varphi, \\
%    y=r\sin\varphi, \\
%    z=z.
%  \end{cases}
%\end{gather*}
%В результате замены переменных задача преобразуется к виду
%\begin{gather}\label{heat:cylindric_coords}
%u_t-a^2\left[\frac1r\frac\partial{\partial r}
%\left(r\frac{\partial u}{\partial r}\right)+
%\frac1{r^2}\frac{\partial^2u}{\partial \varphi^2}+\frac{\partial^2u}{\partial z^2}\right]=0,\\
%\notag
%u|_{t=0}=r^2,\\
%\notag
%\left.\frac{\partial u}{\partial r}\right|_{r=r_0}=0.
%\end{gather}
%Здесь учтено, что нормаль к поверхности цилиндра направлена по радиу\-су.
%
%В задаче (\ref{heat:cylindric_coords}) входные данные (правые части уравнения,
%начального и граничного условий) не зависят от $\varphi$, $z$. Поэтому
%\begin{gather*}
%u=u(r,t),\,\,\,\frac{\partial u}{\partial \varphi}=0,\,\,\,
%\frac{\partial u}{\partial z}=0,
%\end{gather*}
%и задача (\ref{heat:cylindric_coords}) упрощается:
%\begin{gather}
%\label{heat:simplified:cylindric_coords:equation}
%u_t-\frac{a^2}r\frac\partial{\partial r}
%\left(r\frac{\partial u}{\partial r}\right)=0,\,\,\,
%0<r<r_0,\,\,\,t>0,\\
%\label{heat:simplified:cylindric_coords:initial}
%u|_{t=0}=r^2,\,\,\,0\leq r \leq r_0,\\
%\label{heat:simplified:cylindric_coords:boundary}
%u_r|_{r=r_0}=0,\,\,\,t\geq0.
%\end{gather}
%
%Точка $r=0$ является особой точкой
%уравнения (\ref{heat:simplified:cylindric_coords:equation}). Поэтому добавим условие
%ограниченности решения в этой точке в любой момент времени
%\begin{gather}\label{heat:simplified:cylindric_coords:boundness}
%|u|_{r=0}|<\infty,\,\,\,t\geq0.
%\end{gather}
%
%Чтобы решить начально--краевую задачу\label{razdel:perem:1}
%(\ref{heat:simplified:cylindric_coords:equation})--(\ref{heat:simplified:cylindric_coords:boundness}),
%сначала разделим переменные в уравнении (\ref{heat:simplified:cylindric_coords:equation})
%и граничных условиях (\ref{heat:simplified:cylindric_coords:boundary}),
%(\ref{heat:simplified:cylindric_coords:boundness}). С этой целью подставим произведение
%\begin{gather}\label{heat:special_solution}
%u(r,t)=R(r)T(t)
%\end{gather}
%в уравнение (\ref{heat:simplified:cylindric_coords:equation}):
%\begin{gather*}
%T'(t)R(r)=\frac{a^2}rT(t)\frac{d}{dr}
%\left(r\frac{dR(r)}{dr}\right).
%\end{gather*}
%Поделив на $a^2T(t)R(r)$, получим равенство
%\begin{gather}\label{TRequation}
%\frac{T'(t)}{a^2T(t)}=\frac1{rR(r)}\frac{d}{dr}\left(r\frac{dR(r)}{dr}\right),
%\end{gather}
%которое должно выполняться при всех $0<r<r_0$, $t>0$.
%
%Из равенства (\ref{TRequation}) следует, что при фиксированном $t$ правая часть при
%всех $0<r<r_0$ сохраняет постоянное значение и при фиксированном
%$r$  левая часть при всех $t$ также сохраняет постоянное значение.
%Таким образом,
%\begin{gather*}
%\frac{T'(t)}{a^2T(t)}=\frac1{rR(r)}\frac{d}{dr}\left(r\frac{dR(r)}{dr}\right)=-\lambda.
%\end{gather*}
%Отсюда выводим
%\begin{gather}
%\label{heat:Tequation}
%T'(t)+a^2\lambda T(t)=0,\\
%\label{heat:Requation}
%\frac{d}{dr}\left(r\frac{dR(r)}{dr}\right)+\lambda rR(r)=0.
%\end{gather}
%
%
%Разделяя переменные в однородных граничных условиях
% (\ref{heat:simplified:cylindric_coords:boundary}) и (\ref{heat:simplified:cylindric_coords:boundness}),
%будем иметь
%\begin{gather}\label{bcRd}
%|R(0)|<\infty,\,\,\,R'(r_0)=0.
%\end{gather}
%Выпишем собственные числа и собственные функции задачи Штурма--Лиувилля
%(\ref{heat:Requation}), (\ref{bcRd})(\cite[стр.33--34]{BesselMet}).
%Для этой задачи $\lambda_0=0$ является собственным значением, а
%$R_0(r)=1$ --- соответствующей собственной функцией, причём $\|R_0\|^2=\frac{r_0^2}2$.
%Положительные собственные значения определяются положительными нулями функции Бесселя с индексом 1,
%\begin{gather}\label{SLder:lambda}
%\lambda_n=
%\left(\frac{\mu_n}{r_0}\right)^2,\,\,\,J_1(\mu_n)=0,\,\,\,
%n=1,2,\dots,\\
%\notag
%0<\mu_1<\mu_2<\dots<\mu_n<\dots
%\end{gather}
%Соответствующие собственные функции и их нормы имеют вид
%\begin{gather}\label{SLder:R}%\label{SchturmLiouvilleR2:solution}
%R_n(r)=J_0\left(\frac{\mu_nr}{r_0}\right),\\
%\notag
%\|R_n\|^2=\frac{r_0^2}2J_0^2(\mu_n),\,\,\,
%n=1,2,\dots
%\end{gather}
%Подставим собственные значения $\lambda_n$ в уравнение (\ref{heat:Tequation}),
%а решения, соответствующие этим $\lambda_n$, обозначим $T_n(t)$:
%\begin{gather*}
%T_0'(t)=0,\,\,\,T_n'(t)+
%\left(\frac{a\mu_n}{r_0}\right)^2T_n(t)=0,\,\,\,n=1,2,\dots
%\end{gather*}
%Общие решения этих уравнений имеют вид
%\begin{gather*}
%T_0(t)=A_0,\,\,\,T_n(t)=A_ne^{-\left(\frac{a\mu_n}{r_0}\right)^2t},
%\,\,\,n=1,2,\dots
%\end{gather*}
%Согласно формуле  (\ref{heat:special_solution}) запишем частные решения уравнения
%  (\ref{heat:simplified:cylindric_coords:equation}):
%\begin{gather*}
%u_0(r,t)=T_0(t)R_0(r)=A_0,\\
%u_n(r,t)=T_n(t)R_n(r)=A_ne^{-\left(\frac{a\mu_n}{r_0}\right)^2t}
%J_0\left(\frac{\mu_nr}{r_0}\right),
%\,\,\,n=1,2,\dots
%\end{gather*}
%Будем искать решение задачи
%(\ref{heat:simplified:cylindric_coords:equation})--(\ref{heat:simplified:cylindric_coords:boundness})
%в виде ряда
%\begin{gather}\label{heat:u:series}
%u(r,t)=\sum\limits_{n=0}^\infty u_n(r,t)=A_0+\sum\limits_{n=1}^\infty
%A_ne^{-\left(\frac{a\mu_n}{r_0}\right)^2t}J_0\left(\frac{\mu_nr}{r_0}\right).
%\end{gather}
%Функция (\ref{heat:u:series}) удовлетворяет уравнению (\ref{heat:simplified:cylindric_coords:equation})
%при условии равномерной сходимости ряда и возможности почленного дифференцирования
%по $r$ и $t$ до второго порядка включительно. Поскольку каждое слагаемое ряда
% (\ref{heat:u:series}) удовлетворяет граничным условиям
% (\ref{heat:simplified:cylindric_coords:boundary}) и (\ref{heat:simplified:cylindric_coords:boundness}),
%то им же будет удовлетворять и сумма ряда.
%
%Чтобы определить коэффициенты $A_n$, воспользуемся начальным условием
%(\ref{heat:simplified:cylindric_coords:initial}).
%Подставляя ряд (\ref{heat:u:series}) в начальное условие
%(\ref{heat:simplified:cylindric_coords:initial}), получим равенство
%\begin{gather*}
%A_0+\sum\limits_{n=1}^\infty A_nJ_0\left(\frac{\mu_nr}{r_0}\right)=r^2.
%\end{gather*}
%Применяя теорему Стеклова \cite[стр.7]{Fourier1}, заключаем, что
%\begin{gather*}
%A_0=\frac2{r_0^2}\int_0^{r_0}r^3dr=\frac{r_0^2}2,
%\end{gather*}
%\begin{gather*}
%A_n=\frac2{r_0^2J_0^2(\mu_n)}\int_0^{r_0}r^3J_0\left(\frac{\mu_nr}{r_0}\right)dr,\,\,\,
%n=1,2,.
%\end{gather*}
%
%
%
%Вычислим $A_n$, сделав в интеграле замену $\xi=\frac{\mu_nr}{r_0}$
%и воспользовавшись табличным интегралом (\cite[стр.7, формула (3.11)]{BesselMet}):
%\begin{gather*}\label{AnInt}
%A_n=\frac2{r_0^2J_0^2(\mu_n)}\cdot\frac{r_0^4}{\mu_n^4}
%\int_0^{\mu_n}\xi^3J_0(\xi)d\xi
%=\\
%=
%\frac2{J_0^2(\mu_n)}\cdot\frac{r_0^2}{\mu_n^4}
%[2\xi^2J_0(\xi)+(\xi^3-4\xi)J_1(\xi)]|^{\mu_n}_0=\\
%=
%\frac2{J_0^2(\mu_n)}\cdot\frac{r_0^2}{\mu_n^4}\cdot2\mu_n^2J_0(\mu_n)
%=\frac{4r_0^2}{\mu_n^2J_0(\mu_n)}.
%\end{gather*}
%Подставим найденные коэффициенты в ряд (\ref{heat:u:series}) и
%запишем решение задачи:
%\begin{gather*}
%u(r,t)=\frac{r_0^2}2+4r_0^2\sum\limits_{n=1}^\infty
%\frac{e^{-\left(\frac{a\mu_n}{r_0}\right)^2t}}{\mu_n^2J_0(\mu_n)}
%J_0\left(\frac{\mu_nr}{r_0}\right).
%\end{gather*}
%\end{Solution}
%
%\begin{Problem}\label{L23}
%Найти стационарную температуру $u(r,z)$ внутренних точек цилиндра с радиусом
%основания $r_0$ и высотой $h$, если температура нижнего основания зависит только от
%$r$ (расстояния от оси цилиндра), верхнее основание
%теплоизолировано, а боковая поверхность цилиндра свободно охлаждается
%в воздухе нулевой температуры.
%\end{Problem}
%\begin{Solution}
%Запишем однородное  уравнение теплопроводности
%\begin{gather*}%\label{heat:equation:stationary}
%u_t-a^2(u_{xx}+u_{yy}+u_{zz})=0.
%\end{gather*}
%
%Так как температура в каждой точке цилиндра установилась (не
%меняется с течением времени), то
%\begin{gather*}
%    u_t=0
%\end{gather*}
%и уравнение принимает вид
%\begin{gather}\label{heat:equation:stationary}
%u_{xx}+u_{yy}+u_{zz}=0.
%\end{gather}
%
%Поскольку верхнее основание цилиндра теплоизолировано, то
%тепловой поток через верхнее основание равен нулю, то есть
%\begin{gather}\label{heat:boundary:top}
%\left.k\frac{\partial u}{\partial n}\right|_{z=h}=0,
%\end{gather}
%где $n$ --- единичный вектор внешней нормали к верхнему основанию цилиндра.
%
%
%На боковой поверхности цилиндра
%происходит теплообмен с окружающей средой, температура которой равна нулю.
%В соответствии с законом Ньютона граничное условие теплообмена с окружающей средой
%имеет вид (\ref{Newton}). Так как температура внешней среды равна нулю, то
%\begin{gather}\label{heat:boundary:side}
%\left.\left(\frac{\partial u}{\partial n}+Hu\right)\right|_{r=r_0}=0,
%\end{gather}
%где $n$ --- единичный вектор внешней нормали к боковой поверхности цилиндра.
%
%Температура нижнего основания зависит только от
%$r$ (расстояния от оси цилиндра). Соответствующее граничное условие имеет вид
%\begin{gather}\label{heat:boundary:bottom}
%u|_{z=0}=u_0(r),\,\,\,r=\sqrt{x^2+y^2}.
%\end{gather}
%
%Сформулируем краевую задачу. Найти функцию $u(x,y,z)$,
%определённую в области $x^2+y^2\leq r_0^2$, $0\leq z\leq h$,
%удовлетворяющую уравнению (\ref{heat:equation:stationary})  и
%граничным условиям  (\ref{heat:boundary:top})--(\ref{heat:boundary:bottom}).
%
%Область (цилиндр), в которой решается задача,
%ограничена координатными поверхностями $z=0$, $z=h$, $r=r_0$ цилиндрической системы координат.
%В результате перехода к цилиндрическим координатам получим краевую задачу
%\begin{gather}\label{heat:cylindric_coords:stationary}
%\frac1r\frac\partial{\partial r}
%\left(r\frac{\partial u}{\partial r}\right)+
%\frac1{r^2}\frac{\partial^2u}{\partial \varphi^2}+\frac{\partial^2u}{\partial z^2}=0,\\
%\notag
%\left.\left(\frac{\partial u}{\partial r}+Hu\right)\right|_{r=r_0}=0,\,\,\,
%\left.\frac{\partial u}{\partial z}\right|_{z=h}=0,\,\,\,
%u|_{z=0}=u_0(r).
%\end{gather}
%Здесь учтено, что нормаль к боковой поверхности цилиндра направлена по радиусу,
%а нормаль к верхнему основанию направлена вдоль оси $z$.
%
%Так как правые части уравнения и  граничных условий не зависят от $\varphi$, решение задачи не
%зависит от $\varphi$,
%\begin{gather*}
%u=u(r,z),\,\,\,\frac{\partial u}{\partial \varphi}=0.
%\end{gather*}
%Кроме того, в особой точке
%уравнения (\ref{heat:simplified:cylindric_coords:stationary:equation}),
%запишем условие ограниченности решения. Таким образом, $u(r,z)$ должна удовлетворять уравнению
%\begin{gather}
%\label{heat:simplified:cylindric_coords:stationary:equation}
%\frac1r\frac\partial{\partial r}
%\left(r\frac{\partial u}{\partial r}\right)+\frac{\partial^2u}{\partial z^2}=0,\,\,\,0<r<r_0,\,\,\,0<z<h,
%\end{gather}
%и граничным условиям
%\begin{gather}
%\label{heat:simplified:cylindric_coords:stationary:top}
%u_z|_{z=h}=0,\,\,\,0\leq r\leq r_0,\\
%\label{heat:simplified:cylindric_coords:stationary:bottom}
%u|_{z=0}=u_0(r),\,\,\,0\leq r\leq r_0,\\
%\label{heat:simplified:cylindric_coords:stationary:side}
%(u_r+Hu)|_{r=r_0}=0,\,\,\,0\leq z\leq h,\\
%\label{heat:simplified:cylindric_coords:stationary:boundness}
%|u|_{r=0}|<\infty,\,\,\,0\leq z\leq h.
%\end{gather}
%
%Краевую задачу
%(\ref{heat:simplified:cylindric_coords:stationary:equation})--(\ref{heat:simplified:cylindric_coords:stationary:boundness})
%требуется решить внутри прямоугольника, изображённого на рис. \ref{fig1}.
%\begin{figure}[h]
%  % Requires \usepackage{graphicx}
%\centering
%{\unitlength=1mm
%\begin{picture}(70,45)
%\put(0,10){\vector(1,0){60}}
%\put(10,5){\vector(0,1){30}}
%\put(10,30){\line(1,0){40}}
%\put(50,30){\line(0,-1){20}}
%\put(5,27){$h$}
%\put(5,34){$z$}
%\put(50,5){$r_0$}
%\put(60,5){$r$}
%\end{picture}
%}
%  \caption{Область интегрирования уравнения (\ref{heat:simplified:cylindric_coords:stationary:equation})}
%  \label{fig1}
%\end{figure}
%
%
%
%
%В задаче
%(\ref{heat:simplified:cylindric_coords:stationary:equation})--(\ref{heat:simplified:cylindric_coords:stationary:boundness})
%нет начальных условий. В этом случае выбирается пара граничных
%условий на параллельных прямых за основные граничные условия.
%С другой парой граничных условий при решении методом разделения переменных
%поступают как с начальными условиями. В данной задаче условия на сторонах
%прямоугольника $r=0$ и $r=r_0$ являются однородными. Поэтому за основные
%граничные условия возьмём именно их.
%
%\label{urz:razd:per:1}
%Разделим переменные в уравнении (\ref{heat:simplified:cylindric_coords:stationary:equation})
%и граничных условиях (\ref{heat:simplified:cylindric_coords:stationary:side}),
%(\ref{heat:simplified:cylindric_coords:stationary:boundness}). Для этого подставим произведение
%\begin{gather}\label{heat:stationary:special_solution}
%u(r,z)=R(r)Z(z)
%\end{gather}
%в уравнение (\ref{heat:simplified:cylindric_coords:stationary:equation})
%и граничные условия (\ref{heat:simplified:cylindric_coords:stationary:side}),
%(\ref{heat:simplified:cylindric_coords:stationary:boundness}).
%
%\begin{comment}
%\begin{gather*}
%\frac{Z(z)}r\frac{d}{dr}\left(r\frac{dR(r)}{dr}\right)+
%R(r)\frac{d^2Z(z)}{dz^2}=0.
%\end{gather*}
%Поделив на $Z9z)R(r)$, получим равенство
%\begin{gather}\label{ZRequation}
%\frac{1}{rR(r)}\frac{d}{dr}\left(r\frac{dR(r)}{dr}\right)=-
%\frac1{Z(z)}\frac{d^2Z(z)}{dz^2},
%\end{gather}
%которое должно выполняться при всех $0<r<r_0$, $0<z<h$.
%
%Из равенства (\ref{ZRequation}) следует, что при фиксированном $z$ левая часть при
%всех $0<r<r_0$ сохраняет постоянное значение и при фиксированном
%$r$  правая часть при всех $z$ также сохраняет постоянное значение.
%Таким образом,
%\begin{gather*}
%\frac{1}{rR(r)}\frac{d}{dr}\left(r\frac{dR(r)}{dr}\right)=-
%\frac1{Z(z)}\frac{d^2Z(z)}{dz^2}=-\lambda.
%\end{gather*}
%\end{comment}
%В результате получим
%\begin{gather}
%\label{heat:stationary:Zequation}
%Z''(z)-\lambda Z(z)=0,\\
%\label{heat:stationary:Requation}
%\frac{d}{dr}\left(r\frac{dR(r)}{dr}\right)+\lambda rR(r)=0;\\
%\label{bcRd:stationary}
%|R(0)|<\infty,\,\,\,R'(r_0)+HR(r_0)=0.
%\end{gather}
%\label{urz:razd:per:2}
%
%Выпишем собственные числа и собственные функции задачи Штурма--Лиувилля
%(\ref{heat:stationary:Requation}), (\ref{bcRd:stationary})(\cite[стр.34--36]{BesselMet}).
%Собственные числа определяются соотношениями
%\begin{gather*}
%\lambda_n=
%\left(\frac{\mu_n}{r_0}\right)^2,\,\,\,
%n=1,2,\dots,
%\end{gather*}
%где $0<\mu_1<\mu_2<\dots<\mu_n<\dots$ --- положительные корни уравнения
%$J_0'(\mu)\mu+Hr_0J_0(\mu)=0$.
%Соответствующие собственные функции и их нормы имеют вид
%\begin{gather}\label{SchturmLiouvilleR2:solution}
%R_n(r)=J_0\left(\frac{\mu_nr}{r_0}\right),\\
%\notag
%\|R_n\|^2=\frac{r_0^2}2\left[1+\frac{H^2r_0^2}{\mu_n^2}\right]J_0^2(\mu_n),\,\,\,
%n=1,2,\dots
%\end{gather}
%Подставим собственные значения $\lambda_n$ в уравнение (\ref{heat:stationary:Zequation}),
%а решения, соответствующие этим $\lambda_n$, обозначим $Z_n(z)$:
%\begin{gather}\label{ZnEquation}
%Z_n''(z)-
%\left(\frac{\mu_n}{r_0}\right)^2Z_n(z)=0,\,\,\,n=1,2,\dots
%\end{gather}
%Общие решения этих уравнений запишем в виде
%\begin{gather*}
%Z_n(z)=A_n\ch\left(\frac{\mu_n(z-h)}{r_0}\right)+
%B_n\sh\left(\frac{\mu_nz}{r_0}\right),
%\,\,n=1,2,\dots
%\end{gather*}
%Здесь в качестве фундаментальной системы решений уравнения (\ref{ZnEquation})
%использовались $\ch\left(\frac{\mu_n(z-h)}{r_0}\right)$ и
%$\sh\left(\frac{\mu_nz}{r_0}\right)$. Первая функция удовлетворяет условию
%\begin{gather}\label{dZnInh}
%Z_n'(h)=0,
%\end{gather}
%а вторая --- условию
%\begin{gather}\label{ZnIn0}
%Z_n(0)=0.
%\end{gather}
%Условия (\ref{dZnInh}), (\ref{ZnIn0}) --- это однородные условия, соответствующие
%граничным условиям
%(\ref{heat:simplified:cylindric_coords:stationary:top}),
%(\ref{heat:simplified:cylindric_coords:stationary:bottom}).
%При таком выборе фундаментальной системы решений не приходится в дальнейшем\label{rec:parall}
%решать системы уравнений для коэффициентов $A_n$ и $B_n$ и упрощать решение.
%
%Запишем частные решения задачи
%  (\ref{heat:simplified:cylindric_coords:stationary:equation}),
% (\ref{heat:simplified:cylindric_coords:stationary:side}),
%(\ref{heat:simplified:cylindric_coords:stationary:boundness}), имеющие вид
%(\ref{heat:stationary:special_solution}):
%\begin{gather*}
%u_n(r,z)=Z_n(z)R_n(r)=\\
%%\end{gather*}
%%\begin{gather*}
%=
%\left[A_n\ch\left(\frac{\mu_n(z-h)}{r_0}\right)+
%B_n\sh\left(\frac{\mu_nz}{r_0}\right)\right]J_0\left(\frac{\mu_nr}{r_0}\right),\\
%n=1,2,\dots
%\end{gather*}
%В соответствии с принципом дискретной суперпозиции составим новые решения уравнения
%(\ref{heat:simplified:cylindric_coords:stationary:equation}),
%удовлетворяющие условиям
% (\ref{heat:simplified:cylindric_coords:stationary:side}),
%(\ref{heat:simplified:cylindric_coords:stationary:boundness}):
%\begin{gather}\label{heat:u:stationary:series}
%u(r,z)=\sum\limits_{n=1}^\infty u_n(r,z)=\\
%=\notag
%\sum\limits_{n=1}^\infty
%\left[A_n\ch\left(\frac{\mu_n(z-h)}{r_0}\right)+
%B_n\sh\left(\frac{\mu_nz}{r_0}\right)\right]
%J_0\left(\frac{\mu_nr}{r_0}\right).
%\end{gather}
%
%Среди решений (\ref{heat:u:stationary:series}) выберем такое, которое удовлетворяют
%условиям (\ref{heat:simplified:cylindric_coords:stationary:top}),
%(\ref{heat:simplified:cylindric_coords:stationary:bottom}).
%Продифференцируем ряд (\ref{heat:u:stationary:series}) по $z$:
%\begin{gather}\label{heat:u_z:stationary:series}
%u_z(r,z)=
%\sum\limits_{n=1}^\infty
%\frac{\mu_n}{r_0}\biggl[A_n\sh\left(\frac{\mu_n(z-h)}{r_0}\right)+\\
%%\end{gather}
%%\begin{gather*}
%+\notag
%B_n\ch\left(\frac{\mu_nz}{r_0}\right)\biggr]
%J_0\left(\frac{\mu_nr}{r_0}\right).
%\end{gather}
%Подставляя  (\ref{heat:u:stationary:series}) и  (\ref{heat:u_z:stationary:series})
%в граничные условия
% (\ref{heat:simplified:cylindric_coords:stationary:top}) и
%(\ref{heat:simplified:cylindric_coords:stationary:bottom}),
% будем иметь
%\begin{gather}
%\label{AnBnEq1}
%\sum\limits_{n=1}^\infty
%A_n\ch\left(\frac{\mu_nh}{r_0}\right)
%J_0\left(\frac{\mu_nr}{r_0}\right)=u_0(r),\\
%%\end{gather}
%%\begin{gather}
%\label{AnBnEq2}
%\sum\limits_{n=1}^\infty
%\frac{\mu_n}{r_0}B_nJ_0\left(\frac{\mu_nr}{r_0}\right)=0.
%\end{gather}
%Отсюда по формулам \cite[(2.4)]{Fourier1} найдём, что
%\begin{gather}\label{AnBnSteklov}
%A_n\ch\left(\frac{\mu_nh}{r_0}\right)=%\\
%\end{gather}
%\begin{gather}
%=\notag
%\frac{2\mu_n^2}{r_0^2(\mu_n^2+H^2r_0^2)J_0^2(\mu_n)}
%\int\limits_0^{r_0}ru_0(r)J_0\left(\frac{\mu_nr}{r_0}\right)dr\equiv\alpha_n,\\
%\notag
%\frac{\mu_n}{r_0}B_n=0,\,\,\,n=1,2,\dots
%\end{gather}
%
%Подставляя коэффициенты $A_n$ и $B_n$ в ряд (\ref{heat:u:stationary:series}), получим решение задачи:
%\begin{gather*}
%u(r,z)=\sum\limits_{n=1}^\infty
%\frac{\alpha_n\ch\left(\frac{\mu_n(z-h)}{r_0}\right)}{\ch
%\left(\frac{\mu_nh}{r_0}\right)}
%J_0\left(\frac{\mu_nr}{r_0}\right).
%\end{gather*}
%
%\end{Solution}
%
%\section{Метод Фурье для неоднородных уравнений с однородными граничными условиями}\label{NEODN_UR:ODN_GR_USL}
%\begin{Problem}\label{L31}
%Исследовать колебания закреплённой по краю круглой мембраны радиуса $r_0$ с центром в начале координат.
%Колебания вызваны внешней силой, равномерно распределённой с плотностью
%$q(t)=A\sin\omega t$ по площади кольца $r_1<r<r_2$ ($\omega\neq\frac{a\mu_n}{r_0}$,
%где $0<\mu_1<\mu_2<\dots<\mu_n<\dots$ --- положительные нули функции Бесселя $J_0(\mu)$).
%\end{Problem}
%
%\begin{Solution}
%По условию задачи колебания мембраны вызваны внешней силой. Поэтому запишем неоднородное волновое уравнение
%\begin{equation}\label{volnur2Dssoprvn}
%u_{tt}-a^2(u_{xx}+u_{yy})=\frac{f(r,t)}\rho,
%\end{equation}
%где $\rho$ --- поверхностная плотность мембраны, $r=\sqrt{x^2+y^2}$,
%\begin{gather}\label{frt}
%f(r,t)=\begin{cases}
%A\sin\omega t,\mbox{ $r_1<r<r_2$;}\cr
%0,\mbox{ $0<r<r_1$ или $r_2<r<r_0$.}
%\end{cases}
%\end{gather}
%
%
%Поскольку о начальной скорости и начальном отклонении в задаче ничего не сказано, то предполагается, что
%они равны нулю, т.е.
%\begin{gather}\label{nachformamembrvn}
%u|_{t=0}=0,\\
%\label{nacscormembrvn}
%u_t|_{t=0}=0,
%\end{gather}
%и колебания мембраны вызваны только внешней силой.
%
%Кроме начальных условий следует задать граничное условие. Это условие для жёстко закреплённого края
%мембраны имеет вид
%\begin{equation}\label{gruslmembrvn}
%u|_{x^2+y^2=r_0^2}=0.
%\end{equation}
%
%Сформулируем начально--краевую задачу. Найти функцию $u(x,y,t)$, определённую в области
%$x^2+y^2\leq r_0^2$, $t\geq0$, удовлетворяющую уравнению (\ref{volnur2Dssoprvn})
%в области $x^2+y^2< r_0^2$, $t>0$, начальным условиям (\ref{nachformamembrvn}), (\ref{nacscormembrvn}) и граничному условию
%(\ref{gruslmembrvn}).
%
%Будем использовать полярную систему координат, так как мембрана круглая.
%Выпишем уравнение (\ref{volnur2Dssoprvn}), начальные условия
%(\ref{nachformamembrvn}) и (\ref{nacscormembrvn}), и граничное условие (\ref{gruslmembrvn}) в
%полярной системе координат
%\begin{gather}\label{volnur2Dssoprpolsystvn}
%u_{tt}-a^2\left[
%\frac1r
%\frac\partial{\partial r}
%\left(r\frac{\partial u}{\partial r}\right)+
%\frac1{r^2}\frac{\partial^2u}{\partial\varphi^2}\right]=\frac{f(r,t)}{\rho},\\
%\notag
%u|_{t=0}=0,\,\,\,%
%\notag
%u_t|_{t=0}=0,\\
%\notag
%u|_{r=r_0}=0.
%\end{gather}
%
%В силу того, что правая часть уравнения (\ref{volnur2Dssoprpolsystvn}) не зависит от $\varphi$,
% решение задачи также не зависит от $\varphi$,
%\begin{gather*}
%u=u(r,t),\,\,\,\frac{\partial u}{\partial\varphi}=0,
%\end{gather*}
%и задача (\ref{volnur2Dssoprpolsystvn}) упрощается:
%\begin{gather}\label{volnur2Dssoprpolsyst:uprvn}
%u_{tt}-\frac{a^2}r\frac\partial{\partial r}\left(r\frac{\partial u}{\partial r}\right)=
%\frac{f(r,t)}{\rho},\,\,\,0<r<r_0,\,\,\,t>0,\\
%\label{volnur2Dssoprpolsyst:upr:otklvn}
%u|_{t=0}=0,\,\,\,0\leq r\leq r_0,\\
%\label{volnur2Dssoprpolsyst:upr:scorvn}
%u_t|_{t=0}=0,\,\,\,0\leq r\leq r_0,\\
%\label{volnur2Dssoprpolsyst:upr:gruslvn}
%u|_{r=r_0}=0,\,\,\,t\geq0.
%\end{gather}
%
%Для уравнения (\ref{volnur2Dssoprpolsyst:uprvn}) с особой точкой $r=0$ следует добавить условие
%ограниченности решения в этой точке в любой момент времени
%\begin{gather}\label{volnur2Dssoprpolsyst:upr:grusl_ogrvn}
%|u|_{r=0}|<\infty,\,\,\,t\geq0.
%\end{gather}
%
%Решение задачи для неоднородного уравнения с однородными граничными условиями ищется
%в виде ряда по подходящей системе собственных функций.
%Чтобы определить систему собственных функций, подходящую для решения задачи
%(\ref{volnur2Dssoprpolsyst:uprvn})--(\ref{volnur2Dssoprpolsyst:upr:grusl_ogrvn})
% рассмотрим вспомогательную краевую задачу о нахождении нетривиальных решений специального вида
%\begin{equation}\label{spsolutionvn}
%  v(r,t)=R(r)T(t),
%\end{equation}
%однородного уравнения, соответствующего уравнению (\ref{volnur2Dssoprpolsyst:uprvn}),
%\begin{gather}\label{auxprobluravn}
%v_{tt}-\frac{a^2}r\frac\partial{\partial r}\left(r\frac{\partial v}{\partial r}\right)=0,\,\,\,0<r<r_0,
%\,\,\,t>0,
%\end{gather}
%удовлетворяющих граничным условиям (\ref{volnur2Dssoprpolsyst:upr:gruslvn}), (\ref{volnur2Dssoprpolsyst:upr:grusl_ogrvn}),
%\begin{gather}
%\label{auxproblgr}
%v|_{r=r_0}=0,\,\,\,t\geq0,\\
%\label{auxproblcenter}
%|v|_{r=0}|<\infty,\,\,\,t\geq0.
%\end{gather}
%
%Подставляя (\ref{spsolutionvn}) в (\ref{auxprobluravn})--(\ref{auxproblcenter}), будем иметь
%\begin{gather}
%%\label{Tequationvn}
%\notag
%T''(t)+a^2\lambda T(t)=0,\\
%\label{Requationvn}
%\frac{d}{dr}\left(r\frac{dR(r)}{dr}\right)+\lambda rR(r)=0,\\
%\label{bcRvn}
%|R(0)|<\infty,\,\,\,R(r_0)=0.
%\end{gather}
%Задача Штурма--Лиувилля (\ref{Requationvn}), (\ref{bcRvn}) уже использовалась при решении задачи \ref{p21}.
%Её собственные функции, их нормы и собственные числа определяются формулами (\ref{SchturmLiouvilleR:solution}).
%
%Полное решение вспомогательной задачи (\ref{auxprobluravn})--(\ref{auxproblcenter})
%не требуется для решения начально--краевой задачи
%(\ref{volnur2Dssoprpolsyst:uprvn})--(\ref{volnur2Dssoprpolsyst:upr:grusl_ogrvn}). В дальнейшем используются
%только собственные функции (\ref{SchturmLiouvilleR:solution}).
%Поскольку собственные функции образуют полную ортогональную систему,
%решение $u(r,t)$ задачи (\ref{volnur2Dssoprpolsyst:uprvn})--(\ref{volnur2Dssoprpolsyst:upr:grusl_ogrvn})
%представим в виде ряда:
%\begin{gather}\label{urtseries}
%u(r,t)=\sum\limits_{n=1}^\infty C_n(t)R_n(r).
%\end{gather}
%Ряд (\ref{urtseries}) удовлетворяет граничным условиям (\ref{volnur2Dssoprpolsyst:upr:gruslvn}),
%(\ref{volnur2Dssoprpolsyst:upr:grusl_ogrvn}). Подберём функции $C_n(t)$ так, чтобы ряд удовлетворял уравнению
% (\ref{volnur2Dssoprpolsyst:uprvn}) и начальным условиям (\ref{volnur2Dssoprpolsyst:upr:otklvn}),
%(\ref{volnur2Dssoprpolsyst:upr:scorvn}).
%
%Подставив ряд (\ref{urtseries}) в уравнение (\ref{volnur2Dssoprpolsyst:uprvn}), получим
%\begin{gather}\label{volnur2Dssoprpolsyst:uprvn:sumsum}
%\sum\limits_{n=1}^\infty C_n''(t)R_n(r)-a^2\sum\limits_{n=1}^\infty\frac{C_n(t)}{r}\frac{d}{dr}(
%rR_n'(r))=\frac{f(r,t)}{\rho}.
%\end{gather}
%Так как функция $R_n(r)$ является решением уравнения (\ref{Requationvn}), в котором
%$\lambda=\left(\frac{\mu_n}{r_0}\right)^2$, то имеет место тождество
%\begin{gather}\label{Rn:eq:}
%\frac{1}{r}\frac{d}{dr}(rR_n'(r))=-\left(\frac{\mu_n}{r_0}\right)^2R_n(r).
%\end{gather}
%Учитывая (\ref{Rn:eq:}) и используя обозначение $\omega_n=\frac{a\mu_n}{r_0}$,
%преобразуем (\ref{volnur2Dssoprpolsyst:uprvn:sumsum}) к виду
%\begin{gather*}
%\sum\limits_{n=1}^\infty\left[C_n''(t)+\left(\frac{a\mu_n}{r_0}\right)^2C_n(t)\right]R_n(r)=\frac{f(r,t)}{\rho}.
%\end{gather*}
%Данное равенство можно рассматривать как разложение заданной функции $\frac{f(r,t)}{\rho}$ в ряд по
%собственным функциям (\ref{SchturmLiouvilleR:solution}). Коэффициенты разложения с одной стороны выражаются
%через функции $C_n(t)$, с другой определяются формулами \cite[(2.4)]{Fourier1}. Поэтому
%\begin{gather}\label{TnEq}
%C_n''(t)+\omega_n^2C_n(t)=
%\frac1{\|R_n\|^2}\int\limits_0^{r_0}\frac{f(r,t)}{\rho}rR_n(r)dr.
%\end{gather}
%Используя явный вид функции $f(r,t)$ (формула (\ref{frt})), а также собственных функций и их норм
%(формулы  (\ref{SchturmLiouvilleR:solution})), запишем
%\begin{gather}\label{:Cn:eq}
%C_n''(t)+\omega_n^2C_n(t)=\frac{2A\sin\omega t}{r_0^2J_1^2(\mu_n)\rho}
%\int\limits_{r_1}^{r_2}rJ_0\left(\frac{\mu_nr}{r_0}\right)dr.
%\end{gather}
%Вычислим интеграл, стоящий в правой части данного уравнения, сделав замену $z=\frac{\mu_nr}{r_0}$
%и применив формулу \cite[(3.9)]{BesselMet}:\label{r1r2rJ0drINT}
%\begin{gather*}\label{intrJ0mu_nrr0dr}
%\int\limits_{r_1}^{r_2}rJ_0\left(\frac{\mu_nr}{r_0}\right)dr=
%\frac{r_0^2}{\mu_n^2}\int\limits_{\frac{\mu_nr_1}{r_0}}^{\frac{\mu_nr_2}{r_0}}zJ_0(z)dz=\\
%=\frac{r_0}{\mu_n}\biggl[r_2J_1\left(\frac{\mu_nr_2}{r_0}\right)-
%r_1J_1\left(\frac{\mu_nr_1}{r_0}\right)\biggr].
%\end{gather*}
%
%
%Введём обозначение
%\begin{gather}\label{Mn:def}
%M_n=\frac{2A}{J_1^2(\mu_n)\rho\mu_nr_0}\biggl[r_2J_1\left(\frac{\mu_nr_2}{r_0}\right)-
%r_1J_1\left(\frac{\mu_nr_1}{r_0}\right)\biggr]
%\end{gather}
%и перепишем уравнение (\ref{:Cn:eq}) в виде
%\begin{gather}\label{:Cn:eq1}
%С_n''(t)+\omega_n^2C_n(t)=M_n\sin\omega t.
%\end{gather}
%Подставив ряд (\ref{urtseries}) в начальные условия
%(\ref{volnur2Dssoprpolsyst:upr:otklvn}), (\ref{volnur2Dssoprpolsyst:upr:scorvn}), получим, что
%\begin{gather}\label{TnOtkl}
%C_n(0)=0,\\
%\label{TnScor}
%C_n'(0)=0.
%\end{gather}
%
%Запишем общее решение однородного уравнения, соответствующего уравнению (\ref{:Cn:eq1}):
%\begin{gather}\label{TnGsHEq}
%C_n^{оо}(t)=B_n\cos\omega_nt+A_n\sin\omega_nt.
%\end{gather}
%Частное решение $C_n^{чн}$ неоднородного уравнения (\ref{:Cn:eq1}) будем искать в виде
%\begin{gather*}
%C_n^{чн}(t)=D_n\sin\omega t.
%\end{gather*}
%Подставив $C_n^{чн}$ в уравнение (\ref{:Cn:eq1}), получим, что
%\begin{gather*}
%D_n[-\omega^2+\omega_n^2]\sin\omega t=M_n\sin\omega t.
%\end{gather*}
%Отсюда
%\begin{gather*}
%D_n=\frac{M_n}{\omega_n^2-\omega^2}.
%\end{gather*}
%Таким образом,
%\begin{gather}\label{TnPsNHEq}
%C_n^{чн}(t)=\frac{M_n\sin\omega t}{\omega_n^2-\omega^2}.
%\end{gather}
%Общее решение неоднородного уравнения равно сумме общего решения однородного уравнения и
%частного решения (\ref{TnPsNHEq}):
%\begin{gather}\label{TnGsNHEq}
%C_n(t)=B_n\cos\omega_nt+A_n\sin\omega_nt+\frac{M_n\sin\omega t}{\omega_n^2-\omega^2}.
%\end{gather}
%Коэффициенты $A_n$, $B_n$ находим из начальных условий (\ref{TnOtkl}), (\ref{TnScor}).
%Подставляя функцию (\ref{TnGsNHEq}) в условия (\ref{TnOtkl}), (\ref{TnScor}), получим
%\begin{gather}
%\label{Bn:expr}
%B_n=0,\\
%\label{An:expr}
%A_n=-\frac{M_n\omega}{\omega_n(\omega_n^2-\omega^2)}.
%\end{gather}
%
%
%Учитывая (\ref{Bn:expr}), (\ref{An:expr}), из (\ref{TnGsNHEq}) найдём решение задачи
%(\ref{:Cn:eq1})--(\ref{TnScor}):
%\begin{gather*}
%C_n(t)=\frac{M_n}{\omega_n(\omega_n^2-\omega^2)}[\omega_n\sin\omega t-\omega\sin\omega_nt].
%\end{gather*}
%В соответствии с (\ref{urtseries}), запишем теперь ответ задачи:
%\begin{gather}\label{urte}
%u(r,t)=\sum\limits_{n=1}^\infty
%\frac{M_n}{\omega_n(\omega_n^2-\omega^2)}[\omega_n\sin\omega t-\omega\sin\omega_nt]
%J_0\left(\frac{\mu_nr}{r_0}\right).
%\end{gather}
%Здесь коэффициенты $M_n$ определяются формулами (\ref{Mn:def}).
%
%Заметим, что решение (\ref{urte}) можно представить в виде суммы двух рядов. Обе представленные рядами
%функции удовлетворяют однородным граничным условиям
%(\ref{volnur2Dssoprpolsyst:upr:gruslvn}), (\ref{volnur2Dssoprpolsyst:upr:grusl_ogrvn}),
%но различным уравнениям: первая  --- неоднородному уравнению (\ref{volnur2Dssoprpolsyst:uprvn}),
%а вторая --- соответствующему однородному уравнению. Так как первая функция имеет специальный вид
%\begin{gather}\label{up}
%u_ч(r,t)=g(r)\sin\omega t,
%\end{gather}
%%где $g(r)$ --- непрерывно дифференцируемая на интервале $(0,r_0)$ функция.
%то сумму первого ряда, члены которого зависят только от $r$, можно найти в явном виде.
%
%Подставив функцию (\ref{up}) в уравнение (\ref{volnur2Dssoprpolsyst:uprvn}) и в граничные условия
% (\ref{volnur2Dssoprpolsyst:upr:gruslvn}), (\ref{volnur2Dssoprpolsyst:upr:grusl_ogrvn}),
%получим, что $g(r)$ --- решение краевой задачи
%\begin{gather}
%\label{grbp:eq}
%\frac1r\frac{d}{dr}\left(r\frac{dg(r)}{dr}\right)+\left(\frac{\omega}{a}\right)^2g(r)=
%-\frac{A}{a^2\rho}\chi(r),\,\,\,0<r<r_0;\\
%\label{grbp:ogr}
%|g(0)|<\infty;\\
%\label{grbp:bcr0}
%g(r_0)=0.
%\end{gather}
%Здесь через $\chi(r)$ обозначена характеристическая функция отрезка $[r_1,r_2]$, т.е. функция, равная единице
%на отрезке $[r_1,r_2]$, и равная нулю вне этого отрезка. Иными словами, правая часть уравнения
% (\ref{grbp:eq}) задаётся разными выражениями на разных участках интервала $(0,r_0)$.
%
%Общее решение однородного уравнения, соответствующего уравнению (\ref{grbp:eq}), можно записать в виде
%комбинации функций Бесселя и Неймана \cite[(4.2)]{BesselMet}. Решение неоднородного уравнения
%(\ref{grbp:eq}) будем искать методом вариации произвольных постоянных
%\begin{equation}\label{grPrP}
%g(r)=C_1(r)J_0\left(\frac{\omega r}{a}\right)+C_2(r)N_0\left(\frac{\omega r}{a}\right).
%\end{equation}
%Найдём производную:
%\begin{gather}\label{prgrPrP}
%g'(r)=C_1'(r)J_0\left(\frac{\omega r}{a}\right)+C_2'(r)N_0\left(\frac{\omega r}{a}\right)+\\
%+\notag\frac\omega{a}\left(
%C_1(r)J_0'\left(\frac{\omega r}{a}\right)+C_2(r)N_0'\left(\frac{\omega r}{a}\right)\right).
%\end{gather}
%Выберем функции $C_1(r)$ и $C_2(r)$ так, чтобы
%\begin{equation}\label{C1prC2pr}
%C_1'(r)J_0\left(\frac{\omega r}{a}\right)+C_2'(r)N_0\left(\frac{\omega r}{a}\right)=0.
%\end{equation}
%Тогда
%\begin{gather}\label{prgrPrP1}
%g'(r)=\frac\omega{a}\left(
%C_1(r)J_0'\left(\frac{\omega r}{a}\right)+C_2(r)N_0'\left(\frac{\omega r}{a}\right)\right).
%\end{gather}
%
%Вычислим вторую производную $g''(r)$ и подставим $g'(r)$ и $g''(r)$ в уравнение (\ref{grbp:eq}).
%Поскольку $J_0\left(\frac{\omega r}{a}\right)$ и $N_0\left(\frac{\omega r}{a}\right)$ удовлетворяют
%однородному уравнению, то
%\begin{equation}\label{C1prC2prpr}
%\frac\omega{a}(C_1'(r)J_0'\left(\frac{\omega r}{a}\right)+C_2'(r)N_0'\left(\frac{\omega r}{a}\right))=
%-\frac{A}{a^2\rho}\chi(r).
%\end{equation}
%
%Функции $C_1'(r)$ и $C_2'(r)$ определяются из системы (\ref{C1prC2pr}), (\ref{C1prC2prpr}).
%Определитель системы совпадает с определителем Вронского для функций Бесселя и Неймана \cite[стр. 255]{arsenin},
%\begin{gather*}
%\Delta=W\left( J_0\left(\frac{\omega r}{a}\right),N_0\left(\frac{\omega r}{a}\right) \right)=\frac2{\pi r}.
%\end{gather*}
%
%Из системы уравнений (\ref{C1prC2pr}), (\ref{C1prC2prpr}) по формулам Крамера найдём
%\begin{gather}
%\label{C1pr}
%C_1'(r)=\frac{\pi rA}{2a^2\rho}\chi(r)N_0\left(\frac{\omega r}{a}\right),\\
%\label{C2pr}
%C_2'(r)=-\frac{\pi rA}{2a^2\rho}\chi(r)J_0\left(\frac{\omega r}{a}\right).
%\end{gather}
%Сформулируем краевые условия для функций$C_1(r)$ и $C_2(r)$. Из условия ограниченности (\ref{grbp:ogr})
%для функции (\ref{grPrP}) следует, что
%\begin{equation}\label{C2in0}
%C_2(0)=0,
%\end{equation}
%так как функция $N_0\left(\frac{\omega r}{a}\right)$ имеет в точке $r=0$ логарифмическую особенность
%\cite[(4.13)]{BesselMet}. Подставляя (\ref{grPrP}) в граничное условие (\ref{grbp:bcr0}), получим
%\begin{gather}\label{C1C2r0}
%C_1(r_0)J_0\left(\frac{\omega r_0}{a}\right)+C_2(r_0)N_0\left(\frac{\omega r_0}{a}\right)=0.
%\end{gather}
%Отсюда
%\begin{equation}\label{C1inr0}
%C_1(r_0)=-\frac{C_2(r_0)N_0\left(\frac{\omega r_0}{a}\right)}{J_0\left(\frac{\omega r_0}{a}\right)}.
%\end{equation}
%Интегрируя (\ref{C2pr}) по промежутку $[0,r]$, с учётом (\ref{C2in0}) будем иметь
%\begin{gather}\label{C2rExpr}
%C_2(r)=-\frac{\pi A}{2a^2\rho}\int\limits_0^r\xi\chi(\xi)J_0\left(\frac{\omega \xi}{a}\right)d\xi.
%\end{gather}
%Подставляя в интеграл (\ref{C2rExpr}) характеристическую функцию $\chi(\xi)$ и пользуясь свойством аддитивности
%интеграла, получим
%\begin{gather}\label{C2rIntExpr}
%\int\limits_0^r\xi\chi(\xi)J_0\left(\frac{\omega \xi}{a}\right)d\xi=
%    \begin{cases}
%      0,\mbox{ $0\leq r<r_1$;}\cr
%      \int_{r_1}^r\xi J_0\left(\frac{\omega \xi}{a}\right)d\xi,\mbox{ $r_1\leq r<r_2$;}\cr
%      \int_{r_1}^{r_2}\xi J_0\left(\frac{\omega \xi}{a}\right)d\xi,\mbox{ $r_2\leq r\leq r_0$.}
%    \end{cases}
%\end{gather}
%В интегралах (\ref{C2rIntExpr}) сделаем замену $z=\frac\omega{a}\xi$, применим формулу \cite[(3.9)]{BesselMet}
%и введём обозначение
%\begin{gather*}
%\int_{r_1}^{r_2}\xi J_0\left(\frac{\omega \xi}{a}\right)d\xi=\frac a\omega\left[
%r_2J_1\left(\frac{\omega r_2}{a}\right)-r_1J_1\left(\frac{\omega r_1}{a}\right)\right]=\alpha\frac a\omega.
%\end{gather*}
%В результате получим
%\begin{gather}\label{C2ExprFin}
%C_2(r)=
%    \begin{cases}
%      0,\mbox{ $0\leq r<r_1$;}\cr
%      -\frac{\pi A}{2a\omega\rho}
%\left[rJ_1\left(\frac{\omega r}{a}\right)-r_1J_1\left(\frac{\omega r_1}{a}\right)\right],
%\mbox{ $r_1\leq r<r_2$;}\cr
%      -\frac{\pi A}{2a\omega\rho}\alpha,\mbox{ $r_2\leq r\leq r_0$.}
%    \end{cases}
%\end{gather}
%
%Из (\ref{C2ExprFin}) следует, что
%\begin{gather}\label{C2inr0}
%C_2(r_0)=-\frac{\pi A}{2a\omega\rho}\alpha.
%\end{gather}
%Поэтому, в соответствии с (\ref{C1inr0}),
%\begin{equation}\label{C1inr0Fin}
%C_1(r_0)=
%\frac{\pi A\alpha N_0\left(\frac{\omega r_0}{a}\right)}{2a\omega\rho J_0\left(\frac{\omega r_0}{a}\right)}.
%\end{equation}
%Учитывая (\ref{C1inr0Fin}), проинтегрируем (\ref{C1pr}) по промежутку $[r,r_0]$:
%\begin{gather}\label{C1Expr1}
%C_1(r)=-\frac{\pi A}{2a^2\rho}\int\limits_r^{r_0}
%\xi\chi(\xi)N_0\left(\frac{\omega \xi}{a}\right)d\xi+
%\frac{\pi A\alpha N_0\left(\frac{\omega r_0}{a}\right)}{2a\omega\rho J_0\left(\frac{\omega r_0}{a}\right)}.
%\end{gather}
%Интеграл в (\ref{C1Expr1}) вычисляется аналогично интегралу (\ref{C2rIntExpr}). В результате получим
%\begin{gather}\label{C1rIntExpr}
%\int\limits_r^{r_0}\xi\chi(\xi)N_0\left(\frac{\omega \xi}{a}\right)d\xi=
%    \begin{cases}
%      \int_{r_1}^{r_2}\xi J_0\left(\frac{\omega \xi}{a}\right)d\xi,\mbox{ $0\leq r<r_1$;}\cr
%      \int_{r}^{r_2}\xi J_0\left(\frac{\omega \xi}{a}\right)d\xi,\mbox{ $r_1\leq r<r_2$;}\cr
%      0,\mbox{ $r_2\leq r\leq r_0$.}
%    \end{cases}
%\end{gather}
%
%Формула \cite[(3.9)]{BesselMet} справедлива для любой цилиндрической функции. Обозначим
%\begin{gather*}
%\int_{r_1}^{r_2}\xi N_0\left(\frac{\omega \xi}{a}\right)d\xi=\frac a\omega\left[
%r_2N_1\left(\frac{\omega r_2}{a}\right)-r_1N_1\left(\frac{\omega r_1}{a}\right)\right]=\beta\frac a\omega.
%\end{gather*}
%Тогда после вычисления интегралов в (\ref{C1rIntExpr}) и подстановки их в (\ref{C1Expr1}) найдём
%\begin{gather}\label{C1ExprFin}
%C_1(r)=\begin{cases}
%        \frac{\pi A}{2a\omega\rho}\left(-\beta+\alpha
%        \frac{N_0\left(\frac{\omega r_0}{a}\right)}{J_0\left(\frac{\omega r_0}{a}\right)}\right),
%        \mbox{ $0\leq r<r_1$;}\cr
%        \frac{\pi A}{2a\omega\rho}\left(
%        -r_2N_1\left(\frac{\omega r_2}{a}\right)+rN_1\left(\frac{\omega r}{a}\right)+
%        \alpha
%        \frac{N_0\left(\frac{\omega r_0}{a}\right)}{J_0\left(\frac{\omega r_0}{a}\right)}\right),%\cr
%        \mbox{ $r_1\leq r<r_2$;}\cr
%        \frac{\pi A\alpha N_0\left(\frac{\omega r_0}{a}\right)}{2a\omega\rho
%        J_0\left(\frac{\omega r_0}{a}\right)},
%        \mbox{ $r_2\leq r\leq r_0$.}
%       \end{cases}
%\end{gather}
%Осталось подставить функции (\ref{C2ExprFin}), (\ref{C1ExprFin}) в (\ref{grPrP}). В результате получим
%\begin{gather}\label{grExpr}
%g(r)=\begin{cases}
%     \frac{\pi A}{2a\omega\rho}\left(-\beta+\alpha
%        \frac{N_0\left(\frac{\omega r_0}{a}\right)}{J_0\left(\frac{\omega r_0}{a}\right)}\right)
%        J_0\left(\frac{\omega}{a}r\right),
%        \mbox{ $0\leq r<r_1$;}\cr
%     \frac{\pi A}{2a\omega\rho}\biggl\{
%    \left(
%        -r_2N_1\left(\frac{\omega r_2}{a}\right)+\alpha
%        \frac{N_0\left(\frac{\omega r_0}{a}\right)}{J_0\left(\frac{\omega r_0}{a}\right)}\right)
%     J_0\left(\frac{\omega}{a}r\right)+\cr
%     +r_1J_1\left(\frac{\omega r_1}{a}\right)N_0\left(\frac{\omega}{a}r\right)-
%     \frac{2a}{\pi\omega}\biggr\},\mbox{ $r_1\leq r<r_2$;}\cr
%     \frac{\pi A\alpha}{2a\omega\rho}\left(
%     \frac{N_0\left(\frac{\omega r_0}{a}\right)
%     J_0\left(\frac{\omega}{a}r\right)}{J_0\left(\frac{\omega r_0}{a}\right)}-
%     N_0\left(\frac{\omega r}{a}\right)\right),\mbox{ $r_2\leq r\leq r_0$.}
%     \end{cases}
%\end{gather}
%При этом использовалась известная формула \cite[стр. 983]{GradschteinRyzhik}
%$$
%N_1(x)J_0(x)-J_1(x)N_0(x)=-\frac{2}{\pi x}.
%$$
%Решение (\ref{urte}) перепишем в виде
%\begin{gather*}
%u(r,t)=g(r)\sin\omega t-\omega\sum\limits_{n=1}^\infty
%\frac{M_n\sin\omega_nt}{\omega_n(\omega_n^2-\omega^2)}J_0\left(\frac{\mu_nr}{r_0}\right).
%\end{gather*}
%\end{Solution}
%
%
%
%\begin{Problem} \label{L32}
%Рассмотрим задачу \ref{L31} в случае, когда частота $\omega$ внешней силы совпадает с одной
%из собственных частот колебаний. Для определённости, пусть $\omega=\omega_n=\frac{a\mu_n}{r_0}$.
%\end{Problem}
%\begin{Solution}Решение данной задачи может быть получено из формулы (\ref{urte}).
%Для этого  (\ref{urte}) перепишем, изменив переменную суммирования на $k$ и выделив слагаемое с
%индексом $k=n$:
%\begin{gather}\label{urte1}
%u(r,t)=\frac{M_n}{\omega_n(\omega_n^2-\omega^2)}[\omega_n\sin\omega t-\omega\sin\omega_nt]
%J_0\left(\frac{\mu_nr}{r_0}\right)+\\
%\notag
%+\sum\limits_{\substack{k=1\\
%                     k\neq n}}^\infty
%\frac{M_k}{\omega_k(\omega_k^2-\omega^2)}[\omega_k\sin\omega t-\omega\sin\omega_kt]
%J_0\left(\frac{\mu_kr}{r_0}\right).
%\end{gather}
%Решение задачи \ref{L32} находится из (\ref{urte1}) предельным переходом при $\omega\to\omega_n$.
%Вычислим предел первого слагаемого при $\omega\to\omega_n$ по правилу Лопиталя. В результате получим
%\begin{gather*}
%\lim\limits_{\omega\to\omega_n}
%\frac{M_n}{\omega_n(\omega_n^2-\omega^2)}[\omega_n\sin\omega t-\omega\sin\omega_nt]
%J_0\left(\frac{\mu_nr}{r_0}\right)=\\
%=\frac{M_n}{\omega_n}J_0\left(\frac{\mu_nr}{r_0}\right)
%\lim\limits_{\omega\to\omega_n}\frac{t\omega_n\cos\omega t-\sin\omega_nt}{-2\omega}=
%\end{gather*}
%\begin{gather*}
%=\frac{M_n}{2\omega_n}\left(\frac{\sin\omega_nt}{\omega_n}-t\cos\omega_nt\right)
%J_0\left(\frac{\mu_nr}{r_0}\right).
%\end{gather*}
%Второе слагаемое (\ref{urte1}) является функцией, непрерывной при $\omega=\omega_n$, т.к.
%ряд сходится равномерно по $\omega$ при $|\omega-\omega_n|\leq\frac\delta2$, где $\delta=\min\{
%\omega_{n+1}-\omega_n,\omega_n-\omega_{n-1}\}$. Отсюда получаем, что решение задачи
%в случае резонанса имеет вид
%\begin{gather*}
%u(r,t)=\frac{M_n}{2\omega_n}\left(\frac{\sin\omega_nt}{\omega_n}-t\cos\omega_nt\right)
%J_0\left(\frac{\mu_nr}{r_0}\right)+\\
%\notag
%+\sum\limits_{\substack{k=1\\
%                     k\neq n}}^\infty
%\frac{M_k}{\omega_k(\omega_k^2-\omega_n^2)}[\omega_k\sin\omega_n t-\omega_n\sin\omega_kt]
%J_0\left(\frac{\mu_kr}{r_0}\right).
%\end{gather*}
%
%\end{Solution}
%
%
%
%
%
%
%
%\begin{Problem}
%В цилиндре с радиусом основания $r_0$ и высотой $h$ происходит тепловыделение с постоянной плотностью
%$Q$. Найти стационарную температуру $u(r,z)$, если на верхнем основании происходит теплообмен с
%окружающей средой нулевой температуры, нижнее основание поддерживается при постоянной температуре $T_0$,
%а боковая поверхность цилиндра не пропускает тепла.
%\end{Problem}
%\begin{Solution}
%Запишем неоднородное  уравнение теплопроводности
%\begin{gather*}%\label{heat:equation:stationary}
%u_t-a^2(u_{xx}+u_{yy}+u_{zz})=\frac Q{c\rho},
%\end{gather*}
%где $c$ --- коэффициент теплоёмкости материала цилиндра, а $\rho$ --- плотность материала цилиндра.
%
%Так как температура в каждой точке цилиндра установилась (не
%меняется с течением времени), то
%\begin{gather*}
%    u_t=0
%\end{gather*}
%и уравнение принимает вид
%\begin{gather}\label{heat:equation:stationary:rp}
%u_{xx}+u_{yy}+u_{zz}=-\frac Q{c\rho a^2}.
%\end{gather}
%
%Пользуясь условиями задачи, запишем граничные условия:
%на верхнем основании
%\begin{gather}\label{heat:boundary:top:rp}
%\left.\left(\frac{\partial u}{\partial n}+Hu\right)\right|_{z=h}=0;
%\end{gather}
%на нижнем основании
%\begin{gather}\label{heat:boundary:bottom:rp}
%u|_{z=0}=T_0;
%\end{gather}
%на боковой поверхности цилиндра
%\begin{gather}\label{heat:boundary:side:rp}
%\left.k\frac{\partial u}{\partial n}\right|_{r=r_0}=0;
%\end{gather}
%где $n$ --- единичный вектор внешней нормали соответственно к верхнему основанию и к боковой поверхности
%цилиндра.
%
%
%После перехода к цилиндрической системе координат получим уравнение
%\begin{gather}\label{heat:cylindric_coords:stationary:rp}
%\frac1r\frac\partial{\partial r}\left(r\frac{\partial u}{\partial r}\right)+
%\frac1{r^2}\frac{\partial^2u}{\partial \varphi^2}+\frac{\partial^2u}{\partial z^2}=-\frac{Q}{c\rho a^2},%\\
%\end{gather}
%и краевые условия
%\begin{gather}
%\notag
%\left.\left(\frac{\partial u}{\partial z}+Hu\right)\right|_{z=h}=0,\,\,\,
%u|_{z=0}=T_0,\,\,\,\left.\frac{\partial u}{\partial r}\right|_{r=r_0}=0.
%\end{gather}
%
%Поскольку задача обладает осевой симметрией, решение не зависит от $\varphi$,
%\begin{gather*}
%u=u(r,z),\,\,\,\frac{\partial u}{\partial \varphi}=0.
%\end{gather*}
%Поэтому уравнение (\ref{heat:cylindric_coords:stationary:rp}) упрощается, и краевая задача принимает вид
%\begin{gather}
%\label{heat:simplified:cylindric_coords:stationary:equation:rp}
%\frac1r\frac\partial{\partial r}
%\left(r\frac{\partial u}{\partial r}\right)+\frac{\partial^2u}{\partial z^2}=-\frac{Q}{c\rho a^2},\,\,\,
%0<r<r_0,\,\,\,0<z<h,\\
%%\end{gather}
%%\begin{gather}
%\label{heat:simplified:cylindric_coords:stationary:top:rp}
%[u_z+Hu]|_{z=h}=0,\,\,\,0\leq r\leq r_0,\\
%\label{heat:simplified:cylindric_coords:stationary:bottom:rp}
%u|_{z=0}=T_0,\,\,\,0\leq r\leq r_0,\\
%\label{heat:simplified:cylindric_coords:stationary:side:rp}
%u_r|_{r=r_0}=0,\,\,\,0\leq z\leq h,\\
%\label{heat:simplified:cylindric_coords:stationary:boundness:rp}
%|u|_{r=0}|<\infty,\,\,\,0\leq z\leq h.
%\end{gather}
%
%Здесь учтено, что точка $r=0$ является особой точкой
%уравнения (\ref{heat:simplified:cylindric_coords:stationary:equation:rp}).
%
%Краевую задачу
%(\ref{heat:simplified:cylindric_coords:stationary:equation:rp})--(\ref{heat:simplified:cylindric_coords:stationary:boundness:rp})
%требуется решить внутри прямоугольника, изображённого на рис. \ref{fig1}.
%В задаче есть пара однородных граничных условий на параллельных сторонах прямоугольника $r=0$ и $r=r_0$.
%Эти граничные условия будем рассматривать как основные.
%
%Найдём систему собственных функций, которую будем использовать при решении
%краевой задачи
%(\ref{heat:simplified:cylindric_coords:stationary:equation:rp})--(\ref{heat:simplified:cylindric_coords:stationary:boundness:rp}).
%С этой целью разделим переменные в соответствующем однородном уравнении
%\begin{gather}\label{auxprobluravnrp}
%\frac1r\frac\partial{\partial r}
%\left(r\frac{\partial v}{\partial r}\right)+\frac{\partial^2v}{\partial z^2}=0,\,\,\,
%0<r<r_0,\,\,\,0<z<h,%\\
%\end{gather}
%и однородных граничных условиях
%\begin{gather}
%\label{auxproblgrrp}
%v_r|_{r=r_0}=0,\,\,\,0\leq z\leq h,\\
%\label{auxproblcenterrp}
%|v|_{r=0}|<\infty,\,\,\,0\leq z\leq h.
%\end{gather}
%Подставляя
%\begin{equation}\label{spsolutionvnrp}
%  v(r,z)=R(r)Z(z)
%\end{equation}
%в уравнение (\ref{auxprobluravnrp}) и в условия (\ref{auxproblgrrp}), (\ref{auxproblcenterrp}), получим
%\begin{gather}
%\label{heat:stationary:Zequation:rp}
%Z''(z)-\lambda Z(z)=0,\\
%\label{heat:stationary:Requation:rp}
%\frac{d}{dr}\left(r\frac{dR(r)}{dr}\right)+\lambda rR(r)=0;\\
%\label{bcRd:stationary:rp}
%|R(0)|<\infty,\,\,\,R'(r_0)=0.
%\end{gather}
%
%
%Собственные числа и собственные функции задачи Штурма--Лиувилля
%(\ref{heat:stationary:Requation:rp}), (\ref{bcRd:stationary:rp}) выписаны на стр.\pageref{SLder:lambda}
%(формулы (\ref{SLder:lambda}), (\ref{SLder:R})).
%
%Будем искать решение $u(r,z)$ краевой задачи
%(\ref{heat:simplified:cylindric_coords:stationary:equation:rp})--(\ref{heat:simplified:cylindric_coords:stationary:boundness:rp})
%в виде
%\begin{gather}\label{urzseries}
%u(r,z)=C_0(z)R_0(r)+\sum\limits_{n=1}^\infty C_n(z)R_n(r).
%\end{gather}
%Подставим данный ряд в уравнение (\ref{heat:simplified:cylindric_coords:stationary:equation:rp}):
%\begin{gather}\label{urzseries:podst}
%C_0''(z)R_0(r)+\sum\limits_{n=1}^\infty\left[
%\frac{C_n(z)}r\frac{d}{dr}(rR_n'(r))+C_n''(z)R_n(r)\right]=-\frac Q{c\rho a^2}.
%\end{gather}
%Используя тождество
%\begin{gather*}
%\frac1r\frac{d}{dr}(rR_n'(r))\equiv-\left(\frac{\mu_n}{r_0}\right)^2R_n(r)
%\end{gather*}
%для собственной функции $R_n(r)$, равенство (\ref{urzseries:podst}) преобразуем к виду
%\begin{gather*}
%C_0''(z)R_0(r)+\sum\limits_{n=1}^\infty\left[C_n''(z)-\left(\frac{\mu_n}{r_0}\right)^2C_n(z)\right]R_n(r)
%=\notag-\frac Q{c\rho a^2}.
%\end{gather*}
%
%Отсюда, применив теорему Стеклова, получим, что
%\begin{gather*}
%C_0''(z)=-\frac Q{c\rho a^2}\frac1{\|R_0\|^2}\int\limits_0^{r_0}rR_0(r)dr,\\
%C_n''(z)-\left(\frac{\mu_n}{r_0}\right)^2C_n(z)=
%-\frac Q{c\rho a^2}\frac1{\|R_n\|^2}\int\limits_0^{r_0}rR_n(r)dr,\,\,\,n=1,2,\dots
%\end{gather*}
%Вычислим интегралы, стоящие в правых частях полученных уравнений. Прежде всего,
%\begin{gather*}
%-\frac Q{c\rho a^2}\frac1{\|R_0\|^2}\int\limits_0^{r_0}rR_0(r)dr=
%-\frac Q{c\rho a^2}\frac2{r_0^2}\int\limits_0^{r_0}rdr=-\frac Q{c\rho a^2}.
%\end{gather*}
%Второй интеграл перепишем в виде
%\begin{gather*}
%\int\limits_0^{r_0}rR_n(r)dr=\int\limits_0^{r_0}rR_n(r)R_0(r)dr,
%\end{gather*}
%что возможно, поскольку $R_0(r)\equiv1$. Воспользовавшись свойством
%ортогональности различных собственных функций, заключаем, что
%\begin{gather*}
%\int\limits_0^{r_0}rR_n(r)dr=0,\,\,\,n=1,2,\dots
%\end{gather*}
%
%
%
%Таким образом, уравнения на функции $C_0(z)$, $C_n(z)$, $n=1,2,\dots$, принимают вид
%\begin{gather*}
%C_0''(z)=-\frac Q{c\rho a^2},\\
%C_n''(z)-\left(\frac{\mu_n}{r_0}\right)^2C_n(z)=0,\,\,\,n=1,2,\dots
%\end{gather*}
%Запишем общие решения этих уравнений:
%\begin{gather}\label{Z0Expr}
%C_0(z)=-\frac Q{2c\rho a^2}z^2+A_0+B_0z,\\
%\label{ZnExpr}
%C_n(z)=A_n\ch\left(\frac{\mu_nz}{r_0}\right)+B_n\sh\left(\frac{\mu_nz}{r_0}\right),\,\,\,n=1,2,\dots
%\end{gather}
%Подставив ряд (\ref{urzseries}) в граничные условия
%(\ref{heat:simplified:cylindric_coords:stationary:top:rp}) и
%(\ref{heat:simplified:cylindric_coords:stationary:bottom:rp}), получим, что
%\begin{gather*}
%C_0(0)R_0(r)+\sum\limits_{n=1}^\infty C_n(0)R_n(r)=T_0,\\
%(C_0'(h)+HC_0(h))R_0(r)+\sum\limits_{n=1}^\infty(C_n'(h)+HC_n(h))R_n(r)=0.
%\end{gather*}
%Отсюда
%\begin{gather}
%\label{Z0bc}
%C_0(0)=T_0,\,\,\,C_0'(h)+HC_0(h)=0;\\
%\label{Znbc}
%C_n(0)=0,\,\,\,C_n'(h)+HC_n(h)=0.
%\end{gather}
%
%Из граничных условий (\ref{Z0bc}) и (\ref{Znbc}) найдём коэффициенты функций (\ref{Z0Expr}) и (\ref{ZnExpr}):
%\begin{gather*}
%A_0=T_0,\,\,\,B_0=\frac{\frac{Qh(2+Hh)}{2c\rho a^2}-HT_0}{1+Hh},\,\,\,A_n=0,\,\,\,B_n=0.
%\end{gather*}
%С учётом этих коэффициентов,
%\begin{gather}\label{Z0Zne}
%C_0(z)=-\frac Q{2c\rho a^2}z^2+\frac{\frac{Qh(2+Hh)}{2c\rho a^2}-HT_0}{1+Hh}z+T_0,\,\,\,C_n(z)=0.
%\end{gather}
%После подстановки (\ref{Z0Zne}) в ряд (\ref{urzseries}), получим решение задачи
%\begin{gather*}
%    u(r,z)=C_0(z)=-\frac Q{2c\rho a^2}z^2+\frac{\frac{Qh(2+Hh)}{2c\rho a^2}-HT_0}{1+Hh}z+T_0.
%\end{gather*}
%
%
%\end{Solution}
%
%\section{Неоднородные граничные условия}
%\begin{Problem}
%Решить задачу о колебаниях круглой мембраны радиуса $r_0$ с центром в начале координат, к краю которой
%приложена внешняя сила, вызывающая смещение края по закону $A\sin\omega t$, $\omega\neq\frac{a\mu_n}{r_0}$,
%$n=1,2,\dots$, где  $0<\mu_1<\mu_2<\dots<\mu_n<\dots$ --- положительные нули функции $J_0(\mu)$.
%Начальное отклонение и начальная скорость мембраны равны нулю.
%\end{Problem}
%
%\begin{Solution}
%Сформулируем начально--краевую задачу. Найти функцию $u(x,y,t)$, определённую в области
%$x^2+y^2\leq r_0^2$, $t\geq0$, удовлетворяющую уравнению
%\begin{equation}\label{volnur2D}
%u_{tt}-a^2(u_{xx}+u_{yy})=0,
%\end{equation}
%в области $x^2+y^2< r_0^2$, $t>0$, начальным условиям
%\begin{gather}\label{nachformamembr:neodnur}
%u|_{t=0}=0,\\
%\label{nacscormembr:neodnur}
%u_t|_{t=0}=0,
%\end{gather}
%и граничному условию
%\begin{equation}\label{gruslmembr:neodnur}
%u|_{x^2+y^2=r_0^2}= A\sin\omega t.
%\end{equation}
%
%Перейдя в задаче к полярным координатам и учтя осевую симметрию задачи, получим начально--краевую задачу
%\begin{gather}\label{volnur2D:polsyst:upr}
%u_{tt}-\frac{a^2}r\frac\partial{\partial r}\left(r\frac{\partial u}{\partial r}\right)=0,\,\,\,
%0<r<r_0,\,\,\,t>0,\\
%\label{volnur2D:polsyst:upr:otkl}
%u|_{t=0}=0,\,\,\,0\leq r\leq r_0,\\
%\label{volnur2D:polsyst:upr:scor}
%u_t|_{t=0}=0,\,\,\,0\leq r\leq r_0,\\
%\label{volnur2D:polsyst:upr:grusl}
%u|_{r=r_0}=A\sin\omega t,\,\,\,t\geq0,\\
%\label{volnur2D:polsyst:upr:grusl_ogr}
%|u|_{r=0}|<\infty,\,\,\,t\geq0.
%\end{gather}
%Здесь добавлено условие (\ref{volnur2D:polsyst:upr:grusl_ogr}), так как точка $r=0$ является особой точкой
%уравнения (\ref{volnur2D:polsyst:upr}).
%
%Сведём задачу (\ref{volnur2D:polsyst:upr})--(\ref{volnur2D:polsyst:upr:grusl_ogr})
%с неоднородным граничным условием (\ref{volnur2D:polsyst:upr:grusl}) к задаче с однородными граничными
%условиями. Для этого представим решение в виде
%\begin{gather*}
%u(r,t)=v(r,t)+w(r,t),
%\end{gather*}
%где функция $v(r,t)$
%удовлетворяет граничным условиям (\ref{volnur2D:polsyst:upr:grusl}), (\ref{volnur2D:polsyst:upr:grusl_ogr}),
%а $w(r,t)$ --- новая неизвестная функция. Так как существует бесчисленное множество функций, удовлетворяющих
%граничным условиям (\ref{volnur2D:polsyst:upr:grusl}), (\ref{volnur2D:polsyst:upr:grusl_ogr}), то можно
%получить различные представления единственного решения задачи
%(\ref{volnur2D:polsyst:upr})--(\ref{volnur2D:polsyst:upr:grusl_ogr}). В данной задаче удобно функцию $v$
%выбрать удовлетворяющей не только граничным условиям
% (\ref{volnur2D:polsyst:upr:grusl}), (\ref{volnur2D:polsyst:upr:grusl_ogr}),
%но и уравнению (\ref{volnur2D:polsyst:upr}):
%\begin{gather}
%\label{volnur2D:polsyst:upr:v}
%v_{tt}-\frac{a^2}r\frac\partial{\partial r}\left(r\frac{\partial v}{\partial r}\right)=0,\\
%\label{volnur2D:polsyst:upr:grusl:v}
%v|_{r=r_0}=A\sin\omega t,\\
%\label{volnur2D:polsyst:upr:grusl_ogr:v}
%|v|_{r=0}|<\infty.
%\end{gather}
%Среди решений (\ref{volnur2D:polsyst:upr:v})--(\ref{volnur2D:polsyst:upr:grusl_ogr:v})
%выделим функцию специального вида
%\begin{gather}\label{vrtsp}
%v(r,t)=f(r)\sin\omega t.
%\end{gather}
%После подстановки (\ref{vrtsp}) в (\ref{volnur2D:polsyst:upr:v}),  (\ref{volnur2D:polsyst:upr:grusl:v})
%получим краевую задачу для обыкновенного дифференциального уравнения
%\begin{gather}
%\label{volnur2D:polsyst:upr:f}
%\frac{1}{r}\frac{d}{dr}(rf'(r))+\frac{\omega^2}{a^2}f(r)=0,\\
%\label{volnur2D:polsyst:upr:grusl:f}
%f(r_0)=A,\\
%\label{volnur2D:polsyst:upr:grusl_ogr:f}
%|f(0)|<\infty.
%\end{gather}
%Общее решение уравнения (\ref{volnur2D:polsyst:upr:f}) запишем как линейную комбинацию функций Бесселя
%и Неймана
%\begin{gather*}
%f(r)=C_1J_0\left(\frac\omega ar\right)+C_2N_0\left(\frac\omega ar\right).
%\end{gather*}
%Так как функция Неймана неограничена в точке $r=0$, то условие
% (\ref{volnur2D:polsyst:upr:grusl_ogr:f}) выполняется только в случае  $C_2=0$.
%Подставляя функцию
%\begin{gather*}
%f(r)=C_1J_0\left(\frac\omega ar\right)
%\end{gather*}
%в граничное условие (\ref{volnur2D:polsyst:upr:grusl:f}), найдём, что
%\begin{gather*}
%C_1=\frac{A}{J_0\left(\frac\omega ar_0\right)}.
%\end{gather*}
%Таким образом,
%\begin{gather*}
%f(r)=\frac{AJ_0\left(\frac\omega ar\right)}{J_0\left(\frac\omega ar_0\right)},
%\end{gather*}
%а
%\begin{gather}\label{vrte}
%v(r,t)=\frac{AJ_0\left(\frac\omega ar\right)}{J_0\left(\frac\omega ar_0\right)}\sin\omega t.
%\end{gather}
%
%Сформулируем начально--краевую задачу для функции $w(r,t)$.
%Поскольку функции $u(r,t)$ и $v(r,t)$ удовлетворяют однородному уравнению (\ref{volnur2D:polsyst:upr})
%и одним и тем же граничным условиям
% (\ref{volnur2D:polsyst:upr:grusl}),  (\ref{volnur2D:polsyst:upr:grusl_ogr}), то
%\begin{gather*}
%w(r,t)\equiv u(r,t)-v(r,t)
%\end{gather*}
%также  удовлетворяет тому же уравнению,
%\begin{gather}\label{volnur2D:polsyst:upr:w}
%w_{tt}-\frac{a^2}r\frac\partial{\partial r}\left(r\frac{\partial w}{\partial r}\right)=0,\,\,\,
%0<r<r_0,\,\,\,t>0,
%\end{gather}
%но однородным граничным условиям
%\begin{gather}\label{volnur2D:polsyst:upr:grusl:w}
%w|_{r=r_0}=0,\\
%\label{volnur2D:polsyst:upr:grusl_ogr:w}
%|w|_{r=0}|<\infty.
%\end{gather}
%Учитывая (\ref{volnur2D:polsyst:upr:otkl}),  (\ref{volnur2D:polsyst:upr:scor}) и явный вид
%(\ref{vrte}) функции $v(r,t)$, заключаем, что
%\begin{gather}
%\label{volnur2D:polsyst:upr:otkl:w}
%w|_{t=0}=u|_{t=0}-v|_{t=0}=0,\\
%\label{volnur2D:polsyst:upr:otkl:scor:w}
%w_t|_{t=0}=u_t|_{t=0}-v_t|_{t=0}=\frac{-A\omega}{J_0\left(\frac\omega ar_0\right)}
%J_0\left(\frac\omega ar\right).
%\end{gather}
%
%
%Таким образом, решение задачи  (\ref{volnur2D:polsyst:upr})--(\ref{volnur2D:polsyst:upr:otkl:scor:w})
%свелось к решению задачи (\ref{volnur2D:polsyst:upr:w})--(\ref{volnur2D:polsyst:upr:grusl_ogr:w})
%с однородными граничными условиями. В результате разделения переменных в уравнении
%(\ref{volnur2D:polsyst:upr:w}) и граничных условиях (\ref{volnur2D:polsyst:upr:grusl:w}),
%(\ref{volnur2D:polsyst:upr:grusl_ogr:w}) найдём семейство решений (см. стр.\pageref{Requationvn}),
%\begin{gather}\label{wnexpr}
%w_n(r,t)=(A_n\cos\omega_nt+B_n\sin\omega_nt)J_0\left(\frac{\mu_n}{r_0}r\right),\,\,\,n=1,2,\dots,
%\end{gather}
%где $\omega_n=\frac{a\mu_n}{r_0}$.
%
%В соответствии с принципом дискретной суперпозиции из решений уравнения (\ref{volnur2D:polsyst:upr:w})
%составим новое решение
%\begin{gather}\label{wexpr}
%w(r,t)=\sum\limits_{n=1}^\infty w_n(r,t)=
%\sum\limits_{n=1}^\infty(A_n\cos\omega_nt+B_n\sin\omega_nt)J_0\left(\frac{\mu_n}{r_0}r\right).
%\end{gather}
%Подставляя (\ref{wexpr}) в начальные условия (\ref{volnur2D:polsyst:upr:otkl:w}),
%(\ref{volnur2D:polsyst:upr:otkl:scor:w}), будем иметь
%\begin{gather*}
%\sum\limits_{n=1}^\infty A_nJ_0\left(\frac{\mu_n}{r_0}r\right)=0,\\
%\sum\limits_{n=1}^\infty B_n\omega_nJ_0\left(\frac{\mu_n}{r_0}r\right)=
%-\frac{A\omega}{J_0\left(\frac\omega ar_0\right)}J_0\left(\frac\omega ar\right).
%\end{gather*}
%Отсюда
%\begin{gather}\label{AnBnEx}
%A_n=0,\\
%\notag
%B_n=-\frac{2A\omega\int\limits_0^{r_0}rJ_0\left(\frac\omega ar\right)J_0
%\left(\frac{\mu_n}{r_0}r\right)dr}{\omega_nJ_0\left(\frac\omega ar_0\right)r_0^2J_1^2(\mu_n)}.
%\end{gather}
%Для вычисления интеграла воспользуемся формулой \cite[(3.12)]{BesselMet}:
%\begin{gather*}
%\int\limits_0^{r_0}rJ_0\left(\frac\omega ar\right)J_0\left(\frac{\mu_n}{r_0}r\right)dr=
%\left.\frac{\frac{\mu_n}{r_0}rJ_0\left(\frac\omega ar\right)
%J_0'\left(\frac{\mu_n}{r_0}r\right)-
%\frac\omega arJ_0'\left(\frac\omega ar\right)J_0\left(\frac{\mu_n}{r_0}r\right)
%}{\frac{\omega^2}{a^2}-\frac{\mu_n^2}{r_0^2}}\right|_0^{r_0}=\\
%=\frac{a\omega_nr_0J_0\left(\frac\omega ar_0\right)J_1(\mu_n)}{\omega^2-\omega_n^2}.
%\end{gather*}
%Таким образом,
%\begin{gather}\label{BnEx}
%B_n=-\frac{2Aa\omega}{r_0(\omega^2-\omega_n^2)J_1(\mu_n)}.
%\end{gather}
%С учётом (\ref{AnBnEx}),  (\ref{BnEx}) запишем ряд  (\ref{wexpr})
%\begin{gather*}
%w(r,t)=-\frac{2Aa\omega}{r_0}\sum\limits_{n=1}^\infty\frac{J_0
%\left(\frac{\mu_n}{r_0}r\right)\sin\omega_nt}{(\omega^2-\omega_n^2)J_1(\mu_n)}
%\end{gather*}
%и решение задачи (\ref{volnur2D:polsyst:upr})--(\ref{volnur2D:polsyst:upr:grusl_ogr})
%\begin{gather*}
%u(r,t)=\frac{AJ_0\left(\frac\omega ar\right)}{J_0\left(\frac\omega ar_0\right)}\sin\omega t
%-\frac{2Aa\omega}{r_0}\sum\limits_{n=1}^\infty\frac{J_0
%\left(\frac{\mu_n}{r_0}r\right)\sin\omega_nt}{(\omega^2-\omega_n^2)J_1(\mu_n)}.
%\end{gather*}
%\end{Solution}
%
%
%
%
%
%\begin{Problem}
%Определите температуру бесконечного круглого цилиндра радиуса $r_0$, если его начальная температура
%равна нулю, а через боковую поверхность подаётся тепловой поток постоянной плотности $q$.
%\end{Problem}
%
%\begin{Solution}
%Учитывая осевую симметрию задачи, сформулируем её в цилиндрической системе координат:
%\begin{gather}
%\label{heat:simplified:cylindric_coords:equation:1}
%u_t-\frac{a^2}r\frac\partial{\partial r}
%\left(r\frac{\partial u}{\partial r}\right)=0,\,\,\,
%0<r<r_0,\,\,\,t>0,\\
%\label{heat:simplified:cylindric_coords:initial:1}
%u|_{t=0}=0,\,\,\,0\leq r \leq r_0,\\
%\label{heat:simplified:cylindric_coords:boundary:1}
%u_r|_{r=r_0}=\frac qk,\,\,\,t\geq0,\\
%\label{heat:simplified:cylindric_coords:boundness:1}
%|u|_{r=0}|<\infty,\,\,\,t\geq0.
%\end{gather}
%
%Чтобы свести задачу
%(\ref{heat:simplified:cylindric_coords:equation:1})--(\ref{heat:simplified:cylindric_coords:boundness:1})
%к задаче с однородными граничными условиями, представим функцию $u(r,t)$ в виде
%\begin{gather}\label{urepres}
%u(r,t)=v(r,t)+w(r,t),
%\end{gather}
%где функция $v(r,t)$ удовлетворяет граничным условиям (\ref{heat:simplified:cylindric_coords:boundary:1}),
%(\ref{heat:simplified:cylindric_coords:boundness:1}), а $w(r,t)$ --- новая неизвестная функция.
%Простейшие функции, удовлетворяющие граничным условиям (\ref{heat:simplified:cylindric_coords:boundary:1}),
%(\ref{heat:simplified:cylindric_coords:boundness:1}), имеют вид
%\begin{gather}
%\label{vsimplest:1}
%v(r,t)=\frac{q}{k}r,\\
%\label{vsimplest:2}
%v(r,t)=\frac{q}{2kr_0}r^2.
%\end{gather}
%Чтобы сформулировать начально--краевую задачу для $w(r,t)$, следует подставить (\ref{urepres}) в уравнение
%(\ref{heat:simplified:cylindric_coords:equation:1}), начальное условие
%(\ref{heat:simplified:cylindric_coords:initial:1}) и граничные условия
%(\ref{heat:simplified:cylindric_coords:boundary:1}), (\ref{heat:simplified:cylindric_coords:boundness:1}).
%Выбор функции $v(r,t)$ в виде (\ref{vsimplest:1}) приведёт к уравнению с особенностью в правой части,
%поэтому отдадим предпочтение функции (\ref{vsimplest:2}).
%
%
%
%Таким образом,
%\begin{gather}\label{urepres:1}
%u(r,t)=\frac{qr^2}{2kr_0}+w(r,t).
%\end{gather}
%
%Подставляя (\ref{urepres:1}) в уравнение (\ref{heat:simplified:cylindric_coords:equation:1}),
%начальное условие (\ref{heat:simplified:cylindric_coords:initial:1})  и граничные условия
%(\ref{heat:simplified:cylindric_coords:boundary:1}),
%(\ref{heat:simplified:cylindric_coords:boundness:1}), будем иметь
%\begin{gather}\label{v:volnur2D:polsyst:upr:1}
%w_{t}-\frac{a^2}r\frac\partial{\partial r}\left(r\frac{\partial w}{\partial r}\right)=
%\frac{2qa^2}{kr_0},\,\,\,0<r<r_0,\,\,\,t>0,\\
%\label{v:volnur2D:polsyst:upr:otkl:1}
%w|_{t=0}=-\frac{q}{2kr_0}r^2,\,\,\,0\leq r\leq r_0,\\
%\label{v:volnur2D:polsyst:upr:grusl:1}
%w_r|_{r=r_0}=0,\,\,\,t\geq0,\\
%\label{v:volnur2D:polsyst:upr:grusl_ogr:1}
%|w|_{r=0}|<\infty,\,\,\,t\geq0.
%\end{gather}
%
%Решение задачи (\ref{v:volnur2D:polsyst:upr:1})--(\ref{v:volnur2D:polsyst:upr:grusl_ogr:1})
%осуществляется по схеме, предложенной в \S\ref{NEODN_UR:ODN_GR_USL}.
%Чтобы найти подходящую для решения начально--краевой задачи
%(\ref{v:volnur2D:polsyst:upr:1})--(\ref{v:volnur2D:polsyst:upr:grusl_ogr:1}) систему собственных
%функций, следует разделить переменные в соответствующем однородном уравнении и граничных условиях
%(\ref{v:volnur2D:polsyst:upr:grusl:1}), (\ref{v:volnur2D:polsyst:upr:grusl_ogr:1}). Такое разделение
%переменных проведено на стр.\pageref{razdel:perem:1}--\pageref{SLder:R}. Соответствующие собственные
%значения и собственные функции определяются формулами (\ref{SLder:lambda}), (\ref{SLder:R}).
%
%Решение $w(r,t)$ начально--краевой задачи
%(\ref{v:volnur2D:polsyst:upr:1})--(\ref{v:volnur2D:polsyst:upr:grusl_ogr:1}) будем искать в виде ряда
%по системе собственных функций (\ref{SLder:R}):
%\begin{gather}\label{v:urtseries:1}
%w(r,t)=C_0(t)R_0(r)+\sum\limits_{n=1}^\infty C_n(t)R_n(r).
%\end{gather}
%
%Подставив этот ряд в уравнение (\ref{v:volnur2D:polsyst:upr:1}), получим
%\begin{gather}\label{v:volnur2D:polsyst:uprvn:sumsum:1}
%C_0'(t)R_0(r)+\sum\limits_{n=1}^\infty C_n'(t)R_n(r)-\\
%-a^2\sum\limits_{n=1}^\infty\frac{C_n(t)}{r}\frac{d}{dr}(rR_n'(r))\notag=\frac{2qa^2}{kr_0}.
%\end{gather}
%
%
%Учитывая тождество
%\begin{gather}\label{z:Rn:eq:1}
%\frac{1}{r}\frac{d}{dr}(rR_n'(r))=-\left(\frac{\mu_n}{r_0}\right)^2R_n(r)
%\end{gather}
%и используя обозначение $\omega_n=\frac{a\mu_n}{r_0}$, будем иметь
%\begin{gather}\label{C0Cn}
%C_0'(t)R_0(r)+\sum\limits_{n=1}^\infty[C_n'(t)+\omega_n^2C_n(t)]R_n(r)=\frac{2qa^2}{kr_0}.
%\end{gather}
%Так как правую часть равенства (\ref{C0Cn}) можно записать в виде $\frac{2qa^2}{kr_0}=
%\frac{2qa^2}{kr_0}R_0(r)$,  то приравнивая в (\ref{C0Cn}) коэффициенты при одинаковых собственных функциях,
%заключаем, что
%\begin{gather*}%
%C_0'(t)=\frac{2qa^2}{kr_0},\,\,\,C_n'(t)+\omega_n^2C_n(t)=0,\,\,\,n=1,2,\dots
%\end{gather*}
%
%Общие решения этих уравнений имеют вид
%\begin{gather}\label{z:Cn:eq:1}
%C_0(t)=\frac{2qa^2}{kr_0}t+A_0,\,\,\,
%C_n(t)=A_ne^{-\omega_n^2t},\,\,\,n=1,2,\dots
%\end{gather}
%
%Подставив ряд (\ref{v:urtseries:1}) в начальное условие (\ref{v:volnur2D:polsyst:upr:otkl:1})
%получим, что
%\begin{gather*}
%C_0(0)R_0(r)+\sum\limits_{n=1}^\infty C_n(0)R_n(r)=-\frac{qr^2}{2kr_0}.
%\end{gather*}
%
%Отсюда, в силу формул \cite[(2.4)]{Fourier1}, следует, что
%\begin{gather}\label{C0Cn0}
%C_0(0)=-\frac2{r_0^2}\frac{q}{2kr_0}\int\limits_0^{r_0}r^2rdr=-\frac{qr_0}{4k};\\
%\notag
%C_n(0)=-\frac2{r_0^2J_0^2(\mu_n)}\frac{q}{2kr_0}\int\limits_0^{r_0}r^3J_0\left(
%\frac{\mu_nr}{r_0}\right)dr,\,\,\,n=1,2,\dots
%\end{gather}
%Интеграл, стоящий в правой части, был вычислен на стр.\pageref{AnInt}. Таким образом,
%\begin{gather}\label{Cn0ex}
%C_n(0)=-\frac{2qr_0}{\mu_n^2kJ_0(\mu_n)},\,\,\,n=1,2,\dots
%\end{gather}
%Из условий (\ref{C0Cn0}),  (\ref{Cn0ex}) находим коэффициенты функций  (\ref{z:Cn:eq:1}):
%\begin{gather*}
%A_0=-\frac{qr_0}{4k},\,\,\,A_n=-\frac{2qr_0}{\mu_n^2kJ_0(\mu_n)},\,\,\,n=1,2,\dots
%\end{gather*}
%Как следствие,
%\begin{gather*}
%C_0(t)=\frac{2qa^2}{kr_0}t-\frac{qr_0}{4k},\,\,\,
%C_n(t)=-\frac{2qr_0}{\mu_n^2kJ_0(\mu_n)}e^{-\omega_n^2t},\,\,\,n=1,2,\dots
%\end{gather*}
%В соответствии с формулами (\ref{urepres:1}),  (\ref{v:urtseries:1}), ответ задачи запишется так:
%\begin{gather*}
%u(r,t)=\frac{2qa^2}{kr_0}t+\frac{q(2r^2-r_0^2)}{4kr_0}%-\\
%-\frac{2qr_0}{k}\sum\limits_{n=1}^\infty
%\frac{e^{-\omega_n^2t}}{\mu_n^2J_0(\mu_n)}J_0\left(\frac{\mu_nr}{r_0}\right).
%\end{gather*}
%
%\end{Solution}
%
%
%\begin{Problem}
%Найти стационарную температуру $u(r,z)$ внутренних точек цилиндра с радиусом основания $r_0$
%и высотой $h$, если верхнее основание теплоизолированно, через нижнее основание подаётся постоянный
%тепловой поток плотности $q$, а боковая поверхность цилиндра обменивается теплом по закону Ньютона
%с внешней средой постоянной температуры $T_0$.
%\end{Problem}
%\begin{Solution} Стационарная температура удовлетворяет уравнению.
%\begin{gather}\label{StT:NHBRP}
%u_{xx}+u_{yy}+u_{zz}=0.
%\end{gather}
%
%Запишем граничные условия на верхнем основании
%\begin{gather}\label{top}
%\left.k\frac{\partial u}{\partial n}\right|_{z=h}=0;
%\end{gather}
%
%на нижнем основании
%\begin{gather}\label{bottom}
%\left.k\frac{\partial u}{\partial n}\right|_{z=0}=q;
%\end{gather}
%
%на боковой поверхности цилиндра
%\begin{gather}\label{side}
%\left.\left[\frac{\partial u}{\partial n}+H[u-T_0]\right]\right|_{r=r_0}=0;
%\end{gather}
%где $n$ --- единичный вектор внешней нормали соответственно к верхнему основанию, к нижнему основанию, и к
%боковой поверхности цилиндра.
%
%
%После перехода к цилиндрической системе координат с учётом осевой симметрии получим краевую задачу
%\begin{gather}
%\label{StT:NHBRP:cylindric_coords:equation}
%\frac1r\frac\partial{\partial r}
%\left(r\frac{\partial u}{\partial r}\right)+\frac{\partial^2u}{\partial z^2}=0,\,\,\,
%0<r<r_0,\,\,\,0<z<h,\\
%%\end{gather}
%%\begin{gather}
%\label{StT:NHBRP:cylindric_coords:top}
%u_z|_{z=h}=0,\,\,\,0\leq r\leq r_0,\\
%\label{StT:NHBRP:cylindric_coords:bottom}
%u_z|_{z=0}=-\frac qk,\,\,\,0\leq r\leq r_0,\\
%\label{StT:NHBRP:cylindric_coords:side}
%[u_r+Hu]|_{r=r_0}=HT_0,\,\,\,0\leq z\leq h,\\
%\label{StT:NHBRP:cylindric_coords:equation:boundness}
%|u|_{r=0}|<\infty,\,\,\,0\leq z\leq h.
%\end{gather}
%Здесь учтено, что нормаль к боковой поверхности цилиндра направлена по радиусу,
%а нормаль к верхнему и нижнему основаниям направлена параллельно оси $z$.
%
%Уравнение (\ref{StT:NHBRP:cylindric_coords:equation}) требуется решить внутри прямоугольника, изображённого
%на рис.\ref{fig1}. Среди граничных условий
%(\ref{StT:NHBRP:cylindric_coords:top})--(\ref{StT:NHBRP:cylindric_coords:equation:boundness})
%нет пары однородных условий на параллельных прямых. Сведём задачу
%(\ref{StT:NHBRP:cylindric_coords:equation})--(\ref{StT:NHBRP:cylindric_coords:equation:boundness})
%к задаче с однородными граничными условиями при $r=r_0$ и $r=0$.
%
%С этой целью решение краевой задачи
%(\ref{StT:NHBRP:cylindric_coords:equation})--(\ref{StT:NHBRP:cylindric_coords:equation:boundness})
%будем искать в виде
%\begin{gather}\label{u:rep}
%u(r,z)=v(r,z)+w(r,z),
%\end{gather}
%где функция $v(r,z)$ удовлетворяет граничным условиям
%(\ref{StT:NHBRP:cylindric_coords:side}), (\ref{StT:NHBRP:cylindric_coords:equation:boundness}).
%В качестве такой функции возьмём $v(r,z)=T_0$.
%После подстановки (\ref{u:rep}) в
%(\ref{StT:NHBRP:cylindric_coords:equation})--(\ref{StT:NHBRP:cylindric_coords:equation:boundness})
%получим краевую задачу на новую неизвестную функцию $w(r,z)$:
%\begin{gather}\label{v:StT:NHBRP:cylindric_coords:equation}
%\frac1r\frac\partial{\partial r}
%\left(r\frac{\partial w}{\partial r}\right)+\frac{\partial^2w}{\partial z^2}=0,\,\,\,
%0<r<r_0,\,\,\,0<z<h,\\
%%\end{gather}
%%\begin{gather}
%\label{v:StT:NHBRP:cylindric_coords:top}
%w_z|_{z=h}=0,\,\,\,0\leq r\leq r_0,\\
%\label{v:StT:NHBRP:cylindric_coords:bottom}
%w_z|_{z=0}=-\frac qk,\,\,\,0\leq r\leq r_0,\\
%\label{v:StT:NHBRP:cylindric_coords:side}
%[w_r+Hw]|_{r=r_0}=0,\,\,\,0\leq z\leq h,\\
%\label{v:StT:NHBRP:cylindric_coords:equation:boundness}
%|w|_{r=0}|<\infty,\,\,\,0\leq z\leq h.
%\end{gather}
%При решении краевой задачи
%(\ref{v:StT:NHBRP:cylindric_coords:equation})--(\ref{v:StT:NHBRP:cylindric_coords:equation:boundness})
%в качестве основных граничных условий возьмём однородные условия на сторонах прямоугольника $r=0$ и $r=r_0$.
%
%Разделение переменных в уравнении (\ref{v:StT:NHBRP:cylindric_coords:equation}) и граничных условиях
%(\ref{v:StT:NHBRP:cylindric_coords:side}), (\ref{v:StT:NHBRP:cylindric_coords:equation:boundness})
%проведено на стр.\pageref{urz:razd:per:1} при решении аналогичной задачи \ref{L23}. Однако в данной
%задаче граничные условия при $z=0$ и $z=h$ имеют другой тип. Поэтому в соответствии с приведёнными на
%стр.\pageref{rec:parall} рекомендациями относительно выбора фундаментальной системы решений, общее
%решение уравнения (\ref{ZnEquation}) запишем в виде
%\begin{gather*}
%Z_n(z)=A_n\ch\frac{\mu_n(z-h)}{r_0}+B_n\ch\frac{\mu_nz}{r_0}
%\end{gather*}
%Используя частные решения уравнения (\ref{v:StT:NHBRP:cylindric_coords:equation}),
%удовлетворяющие граничным условиям (\ref{v:StT:NHBRP:cylindric_coords:side}),
%(\ref{v:StT:NHBRP:cylindric_coords:equation:boundness}),
%\begin{gather*}
%w_n(r,z)=Z_n(z)J_0\left(\frac{\mu_nr}{r_0}\right),
%\end{gather*}
%составим новое решение
%\begin{gather}\label{WE}
%w(r,z)=\sum\limits_{n=1}^\infty\left[A_n\ch\frac{\mu_n(z-h)}{r_0}+B_n\ch\frac{\mu_nz}{r_0}\right]
%J_0\left(\frac{\mu_nr}{r_0}\right).
%\end{gather}
%Коэффициенты $A_n$ и $B_n$ найдём из граничных условий
%(\ref{v:StT:NHBRP:cylindric_coords:top}), (\ref{v:StT:NHBRP:cylindric_coords:bottom}). После подстановки
%ряда (\ref{WE}) в условия (\ref{v:StT:NHBRP:cylindric_coords:top}),
%(\ref{v:StT:NHBRP:cylindric_coords:bottom}) будем иметь
%\begin{gather*}
%\sum\limits_{n=1}^\infty B_n\frac{\mu_n}{r_0}\sh\frac{\mu_nh}{r_0}J_0\left(\frac{\mu_nr}{r_0}\right)=0,\\
%-\sum\limits_{n=1}^\infty A_n\frac{\mu_n}{r_0}\sh\frac{\mu_nh}{r_0}J_0\left(\frac{\mu_nr}{r_0}\right)=
%\frac{q}{k}.
%\end{gather*}
%Отсюда, по теореме Стеклова \cite[(2.4)]{Fourier1}
%\begin{gather*}
%B_n=0,\,\,\,
%A_n\frac{\mu_n}{r_0}\sh\frac{\mu_nh}{r_0}=
%\frac{-\frac{q}{k}\int\limits_0^{r_0}rJ_0\left(\frac{\mu_nr}{r_0}\right)dr}{\int\limits_0^{r_0}
%rJ_0^2\left(\frac{\mu_nr}{r_0}\right)dr}.
%\end{gather*}
%Учитывая нормы собственных функций (формулы (\ref{SchturmLiouvilleR2:solution})) и значение интеграла
%$\int\limits_0^{r_0}rJ_0\left(\frac{\mu_nr}{r_0}\right)dr=\frac{r_0^2}{\mu_n}J_1(\mu_n)$, вычисленное на
%стр.\pageref{r1r2rJ0drINT}, окончательно получим
%\begin{gather*}
%A_n=-\frac{2qr_0J_1(\mu_n)}{k[\mu_n^2+H^2r_0^2]J_0^2(\mu_n)\sh\frac{\mu_nh}{r_0}}.
%\end{gather*}
%Подставив найденные коэффициенты в ряд (\ref{WE}), запишем ответ задачи
%\begin{gather*}
%u(r,z)=T_0-\frac{2qr_0}{k}\sum\limits_{n=1}^\infty
%\frac{J_1(\mu_n)\ch\frac{\mu_n(z-h)}{r_0}}{\left[
%\mu_n^2+H^2r_0^2\right]J_0^2(\mu_n)\sh\frac{\mu_nh}{r_0}}J_0\left(\frac{\mu_nr}{r_0}\right).
%\end{gather*}
%
%\section{Задачи для самостоятельной работы}
%
%\begin{Problem}
%Найти поперечные колебания круглой мембраны радиуса $r_0$, упруго закреплённой по краю, если известно,
%что в начальный момент времени её форма имеет радиальный характер, а начальная скорость равна нулю.
%\end{Problem}
%
%\begin{Problem}
%Определить температуру в бесконечном круглом цилиндре радиуса $r_0$,
%боковая поверхность которого поддерживается при нулевой температуре, а начальная температура внутри
%цилиндра равна $u(r,0)=A\left(1-\frac{r^2}{r_0^2}\right)$.
%\end{Problem}
%
%\begin{Problem}
%Найти стационарную температуру в однородном цилиндре с радиусом основания $r_0$ и высотой $h$,
%если температуры нижнего и верхнего оснований равны соответственно $T_0$ и
%$T_0\left(1-\frac{r}{r_0}\right)$, а боковая поверхность цилиндра теплоизолированна.
%\end{Problem}
%
%\begin{Problem}
%Круглая закреплённая по краю мембрана радиуса $r_0$ находится в среде, оказывающей сопротивление колебаниям,
%пропорциональное смещению. Коэффициент пропорциональности $\alpha<\left(\frac{a\mu_1}{r_0}\right)^2$,
%где $\mu_1$ --- первый положительный корень уравнения $J_0(\mu)=0$. Рассмотреть колебания мембраны,
%вызванные постоянной силой с плотностью $f_0$.
%\end{Problem}
%
%\begin{Problem}
%Определить температуру бесконечного стержня круглого сечения радиуса $r_0$, внутри которого имеются
%равномерно распределённые тепловые источники постоянной мощности $Q$. На боковой поверхности цилиндра
%происходит теплообмен с окружающей средой нулевой температуры. Начальная температура стержня равна нулю.
%\end{Problem}
%
%\begin{Problem}
%Найти стационарную температуру внутри круглого цилиндра радиуса $r_0$ и высоты $h$,
%если в цилиндре происходит тепловыделение с постоянной плотностью $Q$, на боковой поверхности поддерживается
%нулевая температура, а верхнее и нижнее основания не пропускают тепла.
%\end{Problem}
%
%\begin{Problem}
%К краю круглой мембраны радиуса $r_0$ приложена поперечная сила, пропорциональная $\sin\omega t$,
%$\omega\neq\frac{a\mu_n}{r_0}$, где $\mu_n$ --- положительные нули функции $J_1(\mu)$.
%Найти колебания мембраны, если в начальный момент времени отклонения и скорости всех точек мебраны равны
%нулю.
%\end{Problem}
%
%\begin{Problem}
%Определить температуру бесконечного круглого цилиндра радиуса $r_0$, если его начальная температура
%равна нулю, а на боковой поверхности поддерживается постоянная температура $T_0$.
%\end{Problem}
%
%\begin{Problem}
%Найти стационарную температуру $u(r,z)$ внутренних точек цилиндра с радиусом основания $r_0$ и высотой $h$,
%если через боковую поверхность цилиндра подаётся тепловой поток постоянной плотности $q$, на верхнем
%основании поддерживается постоянная температура $T_0$, а нижнее основание теплоизолировано.
%\end{Problem}
%
%\addcontentsline{toc}{section}{Ответы к задачам для самостоятельной работы}
%\section*{Ответы к задачам для самостоятельной работы}
%
%
%\textbf{5.1.} $u(r,t)=\sum\limits_{n=1}^\infty A_n\cos\omega_ntJ_0\left(\frac{\mu_nr}{r_0}\right)$,
%где $A_n=\frac{2\mu_n^2\int\limits_0^{r_0}rf(r)J_0\left(\frac{\mu_nr}{r_0}\right)dr}{r_0^2
%(\mu_n^2+H^2r_0^2)J_0^2(\mu_n)}$, а $\mu_1<\mu_2<\dots<\mu_n<\dots$ ---
%положительные корни уравнения $J_0'(\mu)\mu+Hr_0J_0(\mu)=0$.
%
%\textbf{5.2.} $u(r,t)=8A\sum\limits_{n=1}^\infty
%\frac{e^{-\left(\frac{a\mu_n}{r_0}\right)^2t}}{\mu_n^3J_1(\mu_n)}J_0\left(\frac{\mu_nr}{r_0}\right)$,
%где $\mu_1<\mu_2<\dots<\mu_n<\dots$ --- положительные корни уравнения $J_0(\mu)=0$.
%
%\textbf{5.3.} $u(r,z)=T_0\left(1-\frac{2z}{3h}+2\sum\limits_{n=1}^\infty
%\frac{\sh\frac{\mu_nz}{r_0}}{\mu_nJ_0(\mu_n)\sh\frac{\mu_nh}{r_0}}J_0\left(\frac{\mu_nr}{r_0}\right)\right)$,
%где $\mu_1<\mu_2<\dots<\mu_n<\dots$ --- положительные корни уравнения $J_1(\mu)=0$.
%
%\textbf{5.4.} $u(r,t)=\frac{f_0}{\rho\alpha}
%\left(\frac{J_0\left(\frac{\sqrt\alpha r}{a}\right)}{J_0\left(\frac{\sqrt\alpha r_0}{a}\right)}-1\right)-
%\frac{2f_0}{\rho}\sum\limits_{n=1}^\infty
%\frac{\cos\sqrt{\omega^2-\alpha}t}{\mu_nJ_1(\mu_n)(\omega^2-\alpha)}J_0\left(\frac{\mu_nr}{r_0}\right)$,
%где $\omega_n=\frac{a\mu_n}{r_0}$, а $\mu_1<\mu_2<\dots<\mu_n<\dots$ --- положительные корни уравнения
%$J_0(\mu)=0$.
%
%\textbf{5.5.} $u(r,t)=\frac{Q}{4c\rho a^2}(2r_0+Hr_0^2-r^2)+\frac{2Q}{c\rho}
%\sum\limits_{n=1}^\infty
%\frac{\mu_nJ_1(\mu_n)
%e^{-\left(\frac{a\mu_n}{r_0}\right)t}}{J_0^2(\mu_n)(\mu_n^2+H^2r_0^2)}J_0\left(\frac{\mu_nr}{r_0}\right)$,
%где $\mu_1<\mu_2<\dots<\mu_n<\dots$ --- положительные корни уравнения $J_0'(\mu)\mu+Hr_0J_0(\mu)=0$.
%
%\textbf{5.6.} $u(r,t)=\frac{Q}{4c\rho a^2}(r_0^2-r^2)$.
%
%\textbf{5.7.} $u(r,t)=\frac{2Aa^2}{r_0k\omega}t-\frac{AaJ_0\left(\frac{\omega r}{a}\right)\sin\omega t}{k
%\omega J_1\left(\frac{\omega r_0}{a}\right)}+\frac{2A\omega}{r_0k}
%\sum\limits_{n=1}^\infty
%\frac{\sin\omega_nt}{\omega_nJ_0(\mu_n)}J_0\left(\frac{\mu_nr}{r_0}\right)$,
%где $\omega_n=\frac{a\mu_n}{r_0}$, а $\mu_1<\mu_2<\dots<\mu_n<\dots$ --- положительные нули функции
%$J_1(\mu)$.
%
%\textbf{5.8.} $u(r,t)=T_0-2T_0\sum\limits_{n=1}^\infty
%\frac{e^{-\left(\frac{a\mu_n}{r_0}\right)^2t}}{\mu_nJ_1(\mu_n)}J_0\left(\frac{\mu_nr}{r_0}\right)$.
%
%\textbf{5.9.} $u(r,t)=\frac{q}{r_0k}(h^2-z^2)+\frac{q}{4r_0k}(2r^2-r_0^2)-
%\frac{2r_0q}{k}\sum\limits_{n=1}^\infty
%\frac{\ch\frac{\mu_nz}{r_0}J_0\left(\frac{\mu_nr}{r_0}\right)}{\mu_n^2\ch\frac{\mu_nh}{r_0}J_0(\mu_n)}$,
%где $\mu_1<\mu_2<\dots<\mu_n<\dots$ --- положительные корни уравнения $J_1(\mu)=0$.
%\begin{comment}
%(формулы (\ref{heat:stationary:special_solution})--(\ref{bcRd:stationary})).
%Собственные числа и собственные функции получающейся задачи Штурма--Лиувилля
%выписаны на с.\pageref{SchturmLiouvilleR2:solution} (формулы (\ref{SchturmLiouvilleR2:solution})).
%
%Будем искать решение $w(r,z)$ краевой задачи
%(\ref{v:StT:NHBRP:cylindric_coords:equation})--(\ref{v:StT:NHBRP:cylindric_coords:equation:boundness})
%в виде
%\begin{gather}\label{vrzseries}
%w(r,z)=\sum\limits_{n=1}^\infty C_n(z)R_n(r).
%\end{gather}
%Подставляя данный ряд в уравнение (\ref{v:StT:NHBRP:cylindric_coords:equation}) и
%используя уравнение для собственных функций, получим, что
%\begin{gather}\label{ZR:sum:1}
%\sum\limits_{n=1}^\infty\left[C_n''(z)-\left(\frac{\mu_n}{r_0}\right)^2C_n(z)\right]R_n(r)=
%-\frac{Q}{c\rho a^2}.
%\end{gather}
%Отсюда
%\begin{gather*}
%C_n''(z)-\left(\frac{\mu_n}{r_0}\right)^2C_n(z)=-\frac{Q}{c\rho a^2\|R_n\|^2}\int\limits_0^{r_0}rR_n(r)dr,\,\,
%n=1,2,\dots
%\end{gather*}
%Учитывая нормы собственных функций (формулы (\ref{SchturmLiouvilleR2:solution})) и значение интеграла $\int\limits_0^{r_0}rJ_0\left(\frac{\mu_nr}{r_0}\right)dr=
%\frac{r^2_0}{\mu_n}J_1(\mu_n)$, вычисленное на стр.\pageref{intrJ0mu_nrr0dr}, перепишем уравнение в виде
%\begin{gather*}
%C_n''(z)-\left(\frac{\mu_n}{r_0}\right)^2C_n(z)=
%-\frac{2QJ_1(\mu_n)}{c\rho a^2\mu_n\left[1+\frac{H^2r_0^2}{\mu_n^2}\right]J_0^2(\mu_n)},\\
%n=1,2,\dots
%\end{gather*}
%Запишем общее решение данного уравнения:
%\begin{gather}\label{CnExpr}
%C_n(z)=A_n\sh\left(\frac{\mu_n(z-h)}{r_0}\right)+B_n\ch\left(\frac{\mu_nz}{r_0}\right)+\\
%%\end{gather}
%%\begin{gather*}
%\notag
%+\frac{2Qr_0^2J_1(\mu_n)}{c\rho a^2\mu_n^3\left[1+\frac{H^2r_0^2}{\mu_n^2}\right]J_0^2(\mu_n)},\,\,\,
%n=1,2,\dots
%\end{gather}
%Подставив ряд (\ref{vrzseries}) в граничные условия
%(\ref{v:StT:NHBRP:cylindric_coords:top}) и (\ref{v:StT:NHBRP:cylindric_coords:bottom}), заключаем, что
%\begin{gather*}
%\sum\limits_{n=1}^\infty C_n(h)R_n(r)=u_0(r)-T_0,\,\,\,
%\sum\limits_{n=1}^\infty C_n'(0)R_n(r)=-\frac qk.
%\end{gather*}
%Применив теорему Стеклова, выводим, что
%\begin{gather}\label{CnBC}
%C_n(h)=\frac{2}{r_0^2\left[1+\frac{H^2r_0^2}{\mu_n^2}\right]J_0^2(\mu_n)}
%\int\limits_0^{r_0}r[u_0(r)-T_0]J_0\left(\frac{\mu_nr}{r_0}\right)dr\equiv\alpha_n,\\
%\notag
%C_n'(0)=\frac{-2qr_0^2J_1(\mu_n)}{k\mu_n\left[1+\frac{H^2r_0^2}{\mu_n^2}\right]J_0^2(\mu_n)}.
%\end{gather}
%Подставив функцию (\ref{CnExpr}) в граничные условия  (\ref{CnBC}), получим, что
%\begin{gather*}
%A_n=-\frac{2qr_0^3J_1(\mu_n)}{k\mu_n^2\ch\left(\frac{\mu_nh}{r_0}\right)
%\left[1+\frac{H^2r_0^2}{\mu_n^2}\right]J_0^2(\mu_n)};\\
%B_n=\frac1{\ch\left(\frac{\mu_nh}{r_0}\right)}\biggl[\alpha_n-
%\frac{2Qr_0^2J_1(\mu_n)}{c\rho a^2\mu_n^3\left[1+\frac{H^2r_0^2}{\mu_n^2}\right]J_0^2(\mu_n)}\biggr].
%\end{gather*}
%Запишем теперь ответ задачи:
%\begin{gather*}
%u(r,z)=
%\sum\limits_{n=1}^\infty\biggl[
%\frac1{\ch\left(\frac{\mu_nh}{r_0}\right)}\biggl[\alpha_n-
%\frac{2Qr_0^2J_1(\mu_n)}{c\rho a^2\mu_n^3\left[1+\frac{H^2r_0^2}{\mu_n^2}\right]J_0^2(\mu_n)}\biggr]
%\ch\left(\frac{\mu_nz}{r_0}\right)-\\
%-\frac{2qr_0^3J_1(\mu_n)}{k\mu_n^2\ch\left(\frac{\mu_nh}{r_0}\right)
%\left[1+\frac{H^2r_0^2}{\mu_n^2}\right]J_0^2(\mu_n)}\sh\left(\frac{\mu_n(z-h)}{r_0}\right)+\\
%+\frac{2Qr_0^2J_1(\mu_n)}{c\rho a^2\mu_n^3\left[1+\frac{H^2r_0^2}{\mu_n^2}\right]J_0^2(\mu_n)}\biggr]
%J_0\left(\frac{\mu_nr}{r_0}\right)+T_0.
%\end{gather*}
%
%\end{comment}
%
%\end{Solution}
%
%\addcontentsline{toc}{section}{Список литературы}
%\begin{thebibliography}{9}
%
%\bibitem{Fourier1} Денисова Н.А. Метод разделения переменных в задачах математической физики:
%Учебно--методическое пособие. --- Н.Новгород: Изд--во ННГУ, 2008. --- 47с.
%
%\bibitem{BesselMet} Гаврилов В.С., Денисова Н.А., Калинин А.В. Цилиндрические функции:
%Учебно--методическое пособие. --- Н.Новгород: Изд--во ННГУ, 2008. --- 42с.
%
%\bibitem{arsenin} Арсенин В.Я. Методы математической физики и специальные функции:  Учебное пособие для
%студентов высших технических учебных заведений. М.: Наука, 1984. --- 383с.
%
%\bibitem{TihonovSamarskij} Тихонов А.Н., Самарский А.А. Уравнения математической физики:
%Учебник для физико--математических специальностей. М.: изд--во Моск. ун--та; Наука, 2004. --- 798с.
%
%\bibitem{BudakSamarskijTihonov} Будак Б.М., Самарский А.А., Тихонов А.Н. Сборник задач по математической
%физике: Учебное пособие для студентов университетов. М.: Наука, 1972. --- 687с.
%
%\bibitem{Vladimirov} Сборник задач по уравнениям математической физики. Под редакцией Владимирова В.С.
%М.: Наука, 1982. --- 254с.
%
%\bibitem{LebedevSkalskayaUflyand} Лебедев Н.Н., Скальская И.П.,  Уфлянд Я.С.
%Сборник задач по математической физике. М.: ГИТТЛ, 1955. --- 420с.
%
%\bibitem{GradschteinRyzhik} Градштейн И.С., Рыжик И.М. Таблицы интегралов, сумм, рядов и произведений. М.:
%гос. изд--во физико--математической литературы, 1963. ---  1097с.
%
%\end{thebibliography}

\end{document}
